\documentclass{llncs} 

\usepackage[final]{listings}
%\usepackage{multicol}
\usepackage{stmaryrd}
\usepackage{amssymb}
%\usepackage{subfig}
%\usepackage{graphicx}
%\usepackage{array}

\lstset{
numbers=left,
numberstyle=\tiny,
language={Java},
mathescape=true,
flexiblecolumns=true,
morekeywords={call,method,var,assert,share,unshare,acquire,release,fork,join,free,invariant,requires,ensures,acc,rd,old},
basicstyle=\sffamily\scriptsize,
moredelim=[is][\itshape]{@}{@},
stepnumber=1,
numbersep=2pt}

\newenvironment{keywords}{
       \list{}{\advance\topsep by0.35cm\relax\small
       \leftmargin=1cm
       \labelwidth=0.35cm
       \listparindent=0.35cm
       \itemindent\listparindent
       \rightmargin\leftmargin}\item[\hskip\labelsep
                                     \bfseries Key words:]}
{\endlist}

% This Source Code Form is subject to the terms of the Mozilla Public
% License, v. 2.0. If a copy of the MPL was not distributed with this
% file, You can obtain one at http://mozilla.org/MPL/2.0/.
%
% Copyright (c) 2011-2019 ETH Zurich.

\newcommand{\Sample}{\ensuremath{\mathsf{Sample}}}

\newcommand{\todo}[1]{\textbf{TODO:#1}}
%\newcommand{\todogiulia}[1]{\textbf{TODO GIULIA: #1}}
%\newcommand{\emphgiulia}[1]{\textbf{#1}}
%\newcommand{\emphgiulia}[1]{#1}
%\newcommand{\newpart}[1]{\hl{#1}}
\newcommand{\goodgap}{\hspace{1.536pt}}


\newcommand{\Java}{\ensuremath{\mathsf{Java}}}
\newcommand{\Scala}{\ensuremath{\mathsf{Scala}}}
\newcommand{\CSharp}{\ensuremath{\mathsf{C\#}}}
\newcommand{\FSharp}{\ensuremath{\mathsf{F\#}}}

\newcommand{\lfp}[2]{\ensuremath{\mathit{lfp}^{#1}_{#2} \hspace{2pt}}}
\newcommand{\parts}[1]{\ensuremath{\wp(#1)}}
\newcommand{\dom}[1]{\ensuremath{\cfunction{dom}(#1)}}
\newcommand{\statement}[1]{\ensuremath{\mathtt{#1}}}
\newcommand{\funzione}[2]{\ensuremath{[#1 \rightarrow#2]}}
\newcommand{\pair}[2]{\ensuremath{#1 \times #2}}
\newcommand{\cartesianproduct}[2]{\ensuremath{#1 \times #2}}
\newcommand{\reducedproduct}[2]{\ensuremath{#1 \varotimes #2}}
\newcommand{\true}{\cel{true}}
\newcommand{\false}{\cel{false}}
\newcommand{\booltop}{\cel{\top_B}}
\newcommand{\reducefunction}{\afunction{red}}
\newcommand{\gettype}{\afunction{getType}}
\newcommand{\getstatictype}[1]{\cfunction{\sttype}(#1)}
\newcommand{\sttype}{statictype}
\newcommand{\abstractn}{abstract}
\newcommand{\abstractfunction}[1]{\afunction{\abstractn}(#1)}


\newcommand{\sequence}[1]{\ensuremath{#1^*}}
\newcommand{\sem}[1]{\ensuremath{\llbracket \mathtt{#1} \rrbracket}}
\newcommand{\semanticanome}[1]{\ensuremath{\mathbb{#1}}}
\newcommand{\semantica}[2]{\ensuremath{\semanticanome{#1}\sem{#2}}}

%Concreto

%Dominio e semantica concrete
\newcommand{\cset}[1]{\ensuremath{\mathsf{#1}}}
\newcommand{\ctrace}[1]{\ensuremath{\trace{{\cset{#1}}}}}
\newcommand{\cel}[1]{\ensuremath{\mathsf{#1}}}
\newcommand{\cjoin}{\ensuremath{\cup}}
\newcommand{\cmeet}{\ensuremath{\cap}}
\newcommand{\corder}{\ensuremath{\subseteq}}
\newcommand{\cbot}{\ensuremath{\emptyset}}
\newcommand{\ctop}[1]{\cset{#1}}
\newcommand{\cfunction}[1]{\ensuremath{\mathit{#1}}}
\newcommand{\csemantics}[2]{\ensuremath{\semantica{#1}{#2}}}

%Astratto

%Dominio e semantica astratte
\newcommand{\aset}[1]{\cset{\overline{#1}}}
\newcommand{\ael}[1]{\cel{\overline{#1}}}
\newcommand{\ajoin}{\ensuremath{\sqcup}}
\newcommand{\ameet}{\ensuremath{\sqcap}}
\newcommand{\awidening}{\ensuremath{\nabla}}
\newcommand{\aorder}{\ensuremath{\sqsubseteq}}
\newcommand{\abot}{\ensuremath{\bot}}
\newcommand{\abtop}{\ensuremath{\top}}
\newcommand{\asemantics}[2]{\semantica{\afunction{\semanticanome{#1}}}{#2}}
\newcommand{\atrace}[1]{\ensuremath{\trace{#1}}}
\newcommand{\afunction}[1]{\ensuremath{\overline{\mathit{#1}}}}

%Stringhe
\newcommand{\alphabet}{\ensuremath{\cset{K}}}
\newcommand{\strings}{\ensuremath{\cset{S}}}
\newcommand{\stringlength}[1]{\ensuremath{\cfunction{len}(#1)}}
\newcommand{\charname}{char}
\newcommand{\cchar}[1]{\ensuremath{\cfunction{\charname}(#1)}}

\newcommand{\cstringsemantics}[2]{\ensuremath{\csemantics{\semanticsstatements}{#1}(#2)}}
\newcommand{\cstringsemanticsboolean}[2]{\ensuremath{\csemantics{\semanticsboolean}{#1}(#2)}}
\newcommand{\semanticsstatements}{S}
\newcommand{\semanticsboolean}{B}



\newcommand{\charsinc}{\ensuremath{\cset{\mathcal{CI}}}}
\newcommand{\acharsinc}{\ensuremath{\aset{\charsinc}}}
\newcommand{\acharsinccontains}{\ensuremath{\aset{C}}}
\newcommand{\acharsincmaybe}{\ensuremath{\aset{MC}}}
\newcommand{\acharsincnotcontains}{\ensuremath{\aset{NC}}}
\newcommand{\acharsincsemantics}[2]{\ensuremath{\asemantics{\semanticsstatements_{\charsinc}}}{#1}(#2)}
\newcommand{\acharsincsemanticsboolean}[2]{\ensuremath{\asemantics{\semanticsboolean_{\charsinc}}}{#1}(#2)}

\newcommand{\prefix}{\ensuremath{\cset{\mathcal{PR}}}}
\newcommand{\aprefix}{\ensuremath{\aset{\prefix}}}
\newcommand{\aprefixsemantics}[2]{\ensuremath{\asemantics{\semanticsstatements_{\prefix}}{#1}(#2)}}
\newcommand{\aprefixsemanticsboolean}[2]{\ensuremath{\asemantics{\semanticsboolean_{\prefix}}{#1}(#2)}}

\newcommand{\suffix}{\ensuremath{\cset{\mathcal{SU}}}}
\newcommand{\asuffix}{\ensuremath{\aset{\suffix}}}
\newcommand{\asuffixsemantics}[2]{\ensuremath{\asemantics{\semanticsstatements_{\suffix}}{#1}(#2)}}
\newcommand{\asuffixsemanticsboolean}[2]{\ensuremath{\asemantics{\semanticsboolean_{\suffix}}{#1}(#2)}}

\newcommand{\asinglebrick}{\ensuremath{\aset{\mathcal{B}}}}
\newcommand{\bricks}{\ensuremath{\cset{\mathcal{BR}}}}
\newcommand{\abricks}{\ensuremath{\aset{\bricks}}}
\newcommand{\abrickssemantics}[2]{\ensuremath{\asemantics{\semanticsstatements_{\bricks}}{#1}(#2)}}
\newcommand{\abrickssemanticsboolean}[2]{\ensuremath{\asemantics{\semanticsboolean_{\bricks}}{#1}(#2)}}
\newcommand{\concatlist}{concatList}
\newcommand{\aconcat}[2]{\ensuremath{\afunction{\concatlist}(#1, #2)}}



\newcommand{\stringgraphs}{\ensuremath{\cset{\mathcal{SG}}}}
\newcommand{\astringgraphs}{\ensuremath{\aset{\stringgraphs}}}
\newcommand{\astringgraphssemantics}[2]{\ensuremath{\asemantics{\semanticsstatements_{\stringgraphs}}{#1}(#2)}}
\newcommand{\astringgraphssemanticsboolean}[2]{\ensuremath{\asemantics{\semanticsboolean_{\stringgraphs}}{#1}(#2)}}

\newcommand{\alabel}[1]{\ensuremath{\afunction{lb}(#1)}}
\newcommand{\aoutdegree}[1]{\ensuremath{\afunction{outdegree}(#1)}}
\newcommand{\aroot}[1]{\ensuremath{\afunction{root}(#1)}}
\newcommand{\anormalizebricks}[1]{\ensuremath{\afunction{normBricks}(#1)}}
\newcommand{\anormalizestringgraphs}[1]{\ensuremath{\afunction{normStringGraph}(#1)}}
\newcommand{\apath}[2]{\ensuremath{\afunction{path}(#1,#2)}}

\pagestyle{plain}

\begin{document}


\title{Automatic Inference of Access Permissions}

\author{Pietro Ferrara \and Peter M\"uller}
\institute{ETH Zurich, Switzerland\\
\email{\{pietro.ferrara, peter.mueller\}@inf.ethz.ch}}

\maketitle

%\todo{What are the key ideas that we should highlight? Is there anything very original?}
\goup
\goup
\goup
\goup
\begin{abstract}
Access permissions are an expressive foundation for the verification
of imperative programs; they simplify framing and provide a
basis for reasoning about concurrent code. They are used in a number
of modern verification approaches such as those based on separation
logic or implicit dynamic frames. However, access permissions increase
the annotation overhead because programmers need to specify for each
program component which permissions it requires from or provides for
its clients. In this paper, we present a new static analysis based on
abstract interpretation to infer access permissions automatically.
Our analysis computes a constraint system that captures the required
access permissions for each heap location at each
program point, which we then solve using linear programming. We 
developed our analysis for the programming language Chalice, which
supports fractional as well as counting permissions. 
Experimental results demonstrate that our analysis is fast and is able
to infer almost all the access permissions for our case studies.
%We deal with Chalice programs, and we infer
%annotation on pre and post conditions, monitor invariants, abstract
%predicates, and loop invariants. 
\end{abstract}
%\begin{keywords}
%Abstract interpretation, static analysis, frame information, access permissions.
%\end{keywords}
\goup
\goup
\goup
\goup
\section{Introduction}
\goup

Verification techniques based on access permissions associate a
permission with each heap location. These permissions can be passed
between methods. A method may access a location if and only if it has
the access permission for that location. This rule simplifies framing
because it implies that a method may modify at most those locations
for which it has permission; all other locations are guaranteed to
remain unchanged. It also enables the verification of concurrent
programs, where permissions allow one to prove the absence of data
races (two threads cannot both have the access permission for any
memory location) and reason about thread interleavings (if a thread
has the permission for a location, no other thread can modify it).
These ideas are for instance used in separation logic
\cite{REY02} and implicit dynamic frames \cite{SJP09}.
Fractional permissions \cite{Boyland03} and counting permissions
\cite{BornatCOP05} refine this permission model by allowing a full
permission to be split (repeatedly) into fractions or into any
number of units, which can be re-composed into a full permission. 
Both fractional and counting permissions allow one to distinguish between
read access (requiring any non-zero permission) and write access
(requiring full permission) and, thus, support parallel reads
while still enforcing exclusive writes.

\begin{figure}
\goup
\goup
\goup
\setlength{\columnseprule}{0.5pt}
\setlength{\columnsep}{10pt}
\begin{multicols}{3}
\begin{lstlisting}[numbers=none]
class Coord {
  var x: int;
  var y: int;

  invariant acc(x) && acc(y)
\end{lstlisting}
\columnbreak
\begin{lstlisting}[numbers=none]
  method client() {
    acquire this;
    var oldX := x; var oldY := y;
    this.FlipH();
    assert x == -oldX;
    assert y == oldY;
    release this;
  }
\end{lstlisting}
\columnbreak
\begin{lstlisting}[numbers=none]
  method FlipH()
  requires acc(x)
  ensures acc(x)
  ensures x == -old(x) 
  { x := -x; }
}
\end{lstlisting}
\end{multicols}
\normalsize
\goup
\caption{A chalice example illustrating access permissions.}
\label{fig:permissionexample}
\goup
\goup
\goup
\end{figure}

\figref{fig:permissionexample} illustrates the use of access
permissions in Chalice \cite{LM09,LMS09}, a programming language and static
program verifier for concurrent class-based programs that supports
both fractional and a flavor of counting permissions. A full
permission to a location \statement{e.f} is denoted by
\statement{acc(e.f)}, which corresponds to
$\statement{e.f}\mapsto\_\;$ in separation logic.  The monitor
invariant of class \statement{Coord} expresses that the monitor holds
the permission to \statement{this.x} and \statement{this.y} when the
monitor is not being held by any thread. When the monitor is being
acquired by a thread, the thread obtains these permissions, and returns
them to the monitor when releasing it. Method \statement{client}
obtains permissions to \statement{this.x} and \statement{this.y} by
acquiring the monitor of \statement{this}. Since it has permission to
\statement{x}, it may call method \statement{FlipH}, which requires
this permission (via its precondition) in order to update the value of
\statement{x} and then returns the permission to the caller (via its
postcondition). Both assertions in \statement{client} verify. The
first assertion verifies because the asserted property is established by the
call to \statement{FlipH}; since the current thread holds the
permission to \statement{x}, no other thread could interfere and
invalidate the property. The second assertion illustrates framing:
since \statement{client} does not pass its permission for \statement{y}
to \statement{FlipH}, we can conclude that \statement{FlipH} is not
allowed to modify \statement{y}. Both assertions would not verify if they
were placed after the \statement{release} statement because then other
threads could obtain permissions to \statement{x} and \statement{y}
and invalidate the asserted properties.

A drawback of all permission systems is that they require programmers
to annotate their programs with access permissions, which increases
the annotation overhead significantly. To address this issue, we
present a new static analysis based on abstract interpretation to
infer access permissions automatically.  We developed our analysis for
Chalice; it infers the access annotations for pre- and postconditions,
monitor invariants, abstract predicates, and loop invariants. Our
paper makes three technical contributions: (1)~a representation of
required fractional and counting permissions with symbolic values,
(2)~the inference of constraints over these symbolic values, whose
solution provides the necessary program annotations, and (3)~an
implementation and experimental evaluation of the analysis in
\Sample\ (Static Analyzer of Multiple Programming LanguagEs).


\goup
\goup
\subsubsection{Approach.}

%This approach is based on a static analysis based on the abstract interpretation theory \cite{CC77,CC79}, and it supports several Chalice features (namely, predicates, sharing of variables, monitors, fork and join of threads). The analysis relies on a heap abstraction that soundly approximates all the concrete structures of the heap.

Our approach is based on abstract interpretation \cite{CC77,CC79}, a
theory for defining and soundly approximating the semantics of a
program. Then the main components of our analysis are: (i) an \emph{abstract domain} that is a complete
lattice $\langle \aset{A}, \aorder_\aset{A}, \abot_\aset{A},
\abtop_\aset{A}, \ajoin_\aset{A}, \ameet_\aset{A} \rangle$, (ii) a
\emph{widening} operator to make  the analysis convergent, and (iii) an
\emph{abstract semantics} defined as a transfer function that, given a
statement and an initial abstract state, defines the abstract state
obtained after the given statement.

We introduce a \emph{symbolic value} for each location and each
possible occurrence of an access permission in a
pre- or postcondition, monitor invariant, or abstract
predicate.  For instance, \presv{C}{m}{x.f} represents the access
permission specified in the precondition of method \statement{m} of
class \statement{C} for the location denoted by the path
\statement{x.f}.  Using these symbolic values, the analysis infers a
sound approximation of the access permissions that the current thread
has for any given heap location at any given program
point. These \emph{symbolic permissions} have the form of $\sum a_i *
\statement{v}_i + c$ where $a_i$ and $c$ are integers in
\integernumbers, and $\statement{v}_i$ is a symbolic value. For
instance, if method \statement{m}'s first statement acquires the
monitor of \statement{this} then its symbolic permission \statement{l}
for a location \statement{x.f} is $1 * \presv{C}{m}{x.f} + 1 *
\misv{C}{x.f} + 0$.

We then extract a set of \emph{constraints} over the symbolic values,
which reflect the permission rules of Chalice. For instance, if
\statement{m}'s next statement is the field write
$\statement{x.f}:=\ldots$, we introduce a constraint that the symbolic
permission \statement{l} at this program point is equal to the full (write)
permission. We solve the resulting constraint system using linear
programming and obtain a numerical access permission for each symbolic
value. For simplicity, we assume here that the inference is run on
un-annotated programs, but partial annotations could be easily represented as additional constraints.


%% In addition, we can introduce some optional unsoundness when building
%% up the entry state of a method, and when inhaling permissions on
%% summary locations. We adopt this unsoundness in order to increase the
%% amount of inferred permissions, and this is particularly useful when
%% the heap approximation is too coarse.

%The approach has been \emph{implemented} in \Sample\ (Static Analyzer of Multiple Programming LanguagEs). 
%The structure of \Sample\ is such that it can be plugged with different domains, e.g., heap analyses. \Sample\ works on an intermediate language called \Simple, and we have already implemented the translation from \Scala\ to this language. Thus the examples we analyze are written in \Scala, and we add a library to simulate Chalice primitives. 

The experimental results show that our analysis is fast and
accurate. For all examples in our test suite, we infer permission
annotations in under three seconds. Our analysis infers all
necessary permissions in all examples except for two that use
recursive data structures, for which our heap abstraction is too
coarse. We expect more precise heap analyses like TVLA \cite{SRW02}
to solve this issue.




\goup
\goup
\subsubsection{Outline.}

\secref{sect:language} introduces the language supported by our
analysis and the running example. \secref{sect:heapanalysis} sketches
our heap analysis, while \secref{sect:abstractdomain} defines the
symbolic abstract domain and semantics to approximate access
permissions. \secref{sect:inference} presents how we infer access
permissions.  \secref{sect:experimentalresults} reports the
experimental results.  \secref{sect:relatedwork} discusses related
work, and \secref{sect:conclusion} concludes.




%\noindent \textbf{Generic analyzers} Abstract interpretation can be applied in order to develop generic analyzers \cite{COU99}. In particular, this theory allows one to define a compositional analysis, e.g., an analysis that can be instantiated with different numerical domains, and in order to analyze different properties. Many different generic analyzers have been proposed recently \cite{FER09a,LF08,SPO05}. The analysis we present in this paper has been implemented in \Sample.
\goup
\section{Language}
\label{sect:language}

In this section, we introduce the subset of Chalice used
in the rest of the paper. For brevity, we omit abstract predicates
here, but our tool handles them.

A program consists of a sequence of class declarations.  Each class
declares fields, a monitor invariant, and methods. At runtime, each
object has a (non-reentrant) monitor.  A method declaration contains
the method signature, pre- and postconditions, and a method body,
which is given as a Control Flow Graph (cfg) of basic blocks. Each
basic block consists of a sequence of statements. Different blocks are
connected through edges that optionally contain a boolean condition to
represent conditional jumps, for instance, to represent conditionals
and loops. Specifications are expressed using the expressions of the programming
language, an \statement{old}-expression to let postconditions refer to
prestate-values, and permission predicates, which we describe below.


\goup
\goup
\subsubsection{Expressions and Statements.}

The expressions and statements of our language are summarized in 
Table~\ref{fig:language}. The expressions are local variable 
access and field read; we omit uninteresting expressions such as boolean and arithmetic operators here, but use them in our examples.

\begin{table}[ht]
\begin{center}
\statement{E ::=} \statement{y} \texttt{|} \statement{x.f}\\% \texttt{|} \statement{new\ T}\\
\begin{tabular}{rllllllll}
\statement{St\ ::} & \texttt{=}\ 
\statement{x:=E} & 
\texttt{|} \statement{x:=new\ T} &
\texttt{|} \statement{acquire\ x} &
\texttt{|} \statement{t:=fork\ x.m()} &
\texttt{|} \statement{share\ x} & 
%\texttt{|} \statement{fold\ x.pred} &
\\
&  
\texttt{|} \statement{x.f:=E} &
\texttt{|} \statement{x.m()}  &
%\texttt{|} \statement{free\ x}&
\texttt{|} \statement{release\ x} &
\texttt{|} \statement{join\ t} 
%\texttt{|} \statement{unshare\ x} &
%\texttt{|} \statement{unfold\ x.pred} 
\end{tabular}
\goup
\end{center}
\caption{Expressions and statements of our Chalice subset.}
\label{fig:language}
\end{table}

The statements of our language are: local variable assignment, field
write, object creation, method call, acquiring and releasing monitors,
forking and joining threads, as well as sharing a previously
thread-local object, that is, making the object available for locking.

\goup
\goup
\subsubsection{Permissions.}

\lstset{basicstyle=\sffamily\normalsize}
A permission in Chalice has the form $p\% + n\cdot\epsilon$, where $p$
is a natural number between 0 and 100, $n$ is an integer, and
$\epsilon$ is an infinitesimal permission, that is, an arbitrarily
small, positive number. The percentage encodes fractional permissions,
whereas the infinitesimal permission is used as counting permissions.
A thread may access a heap location if and only if it has the
permission to do that. A value strictly greater than 0 allows a thread
to read (that is, $p > 0$ or $n>0$), but only a full 100\% permission
($p=100$ and $n=0$) allows a thread to write. In specifications, we
write \statement{acc(x.f,p)} and \statement{rd(x.f)} to denote a
\statement{p}\% permission and an infinitesimal permission for
location \statement{x.f}, respectively. A full permission is written
as \statement{acc(x.f)}, which abbreviates \statement{acc(x.f,100)}.
Specification must be \emph{self-framing}, that is, a specification may refer to
\statement{x.f} only if it also includes at least a read permission
for \statement{x.f}. For instance, \lstinline{rd(x.f) && x.f > 0} is
a valid specification, but  \lstinline{x.f > 0} is not. 

\goup
\goup
\subsubsection{Permission Transfer.}

\lstset{basicstyle=\sffamily\scriptsize} 
Permissions are held by
threads, more precisely by a method incarnation. Moreover, they can be stored in monitors. The semantics of many statements includes a permission transfer between these
places. This transfer is encoded via two basic operations. The
\emph{exhale} operation takes a specification $s$, proves that $s$
holds, and consumes the permissions described by $s$; that is, it
checks that the current thread has the required permissions and then
removes them. The \emph{inhale} operation takes a specification $s$,
assumes that $s$ holds, and provides permissions described by $s$ to
the current thread; that is, it adds permissions and, if the thread
previously had no permission for a location, assigns an arbitrary
value to that location to reflect the possible changes that other
threads might have made while the current thread had no permission.

The statements of our language transfer permissions as follows.
Assignments contain no permission transfer.  Object creation inhales
full access to all locations of the fresh object.  A method call
exhales the precondition of the callee and then inhales its
postcondition.  When a thread acquires a monitor, it inhales the
monitor invariant, while it exhales the invariant when it releases the
monitor. Fork exhales the precondition of the forked method, while
join inhales the postcondition.  Sharing an object exhales its monitor invariant;
threads can then acquire the monitor to obtain the permissions
described by the invariant.

The constraints collected by our static analysis depend on the
requirements a statement makes on the thread's permissions as well as
on the permission transfer the statement performs. It is therefore
straightforward to extend our analysis to statements no presented here,
such as de-allocation or message passing \cite{LM10}.





%We identified three main aspects of the language: (i) heaps modification (handled by an external heap analysis - Section \ref{sect:heapanalysis}), (ii) inhaling and exhaling of permissions (handled by our abstract domain and semantics - Section \ref{sect:abstractdomain}), and (iii) constraint inference (handled using the information tracked by the abstract semantics and imposing some constraints on it - Section \ref{sect:inference}). For this reason, we expect that our approach can be straightforwardly extended to support other features, like channels .


\goup
\goup
\subsubsection{Running Example.}
\label{sect:runningexample}

\begin{figure}
\goup
\goup
\goup
\setlength{\columnseprule}{0.5pt}
\setlength{\columnsep}{20pt}
\begin{multicols}{3}
\begin{lstlisting}
class W1 {
  var c : Cell;
  method Inc()
  ensures c.c1==old(c.c1)+1
  {
    acquire c;
    c.x := c.x+1;
    c.c1 := c.c1+1;
    release c;
  }
}
\end{lstlisting}
\begin{lstlisting}[firstnumber=12]
class Cell {
  var x, c1, c2: int;
  invariant x==c1+c2;
}
\end{lstlisting}
\columnbreak
\begin{lstlisting}[firstnumber=16]
class W2 {
  var c : Cell;
  method Inc()
  ensures c.c2==old(c.c2)+1
  {
    acquire c;
    c.x := c.x+1;
    c.c2 := c.c2+1;
    release c;
  }
}
\end{lstlisting}
\columnbreak
\begin{lstlisting}[firstnumber=27]
class OwickiGries {
  method main() {
    var c := new Cell;
    share c;
    var w1 := new W1;
    w1.c := c;
    var w2 := new W2;
    w2.c = c;
    t1 := fork w1.Inc();
    t2 := fork w2.Inc();
    join t1;
    join t2;
    acquire c;
    assert c.x==2;
  }
}
\end{lstlisting}
\end{multicols}
\normalsize
\goup
\caption{The Owicki-Gries example without access specifications.}
\label{fig:runningexample}
\goup
\goup
\goup
\end{figure}

We illustrate our inference using Owicki and Gries's classical example
\cite{OG76}, see \figref{fig:runningexample}. Two worker threads,
implemented in classes \statement{W1} and \statement{W2} each increment
a shared variable \statement{x} by 1. The client (method \statement{main})
asserts that the effect of running both workers is to increment 
\statement{x} by 2. Proving this assertion requires ghost state; the
ghost fields \statement{c1} and \statement{c2} store the contribution
of each worker to the overall effect. These ghost variables are related
to \statement{x} in \statement{Cell}'s monitor invariant and also
mentioned in the workers' postconditions. This design requires fractional
permissions. Putting full permission to the ghost variables into the
monitor invariant would prevent the workers from mentioning them in
the postconditions because they would then not be self-framing. Conversely,
putting full permissions into the postconditions prevents the monitor
invariant from relating the ghost fields \statement{c1} and \statement{c2} to \statement{x}. 

When splitting the permissions over the postconditions and the monitor
invariant, the example verifies. From the postconditions, we know that
after the two join operations in \statement{main}, \statement{c1} and
\statement{c2} have each been increased by 1. From \statement{Cell}'s
monitor invariant, which we can assume after the acquire statement, we
know that then \statement{x} has increased by 2. We will show how our
analysis infers the permission annotation that enables this
verification.


\goup
\section{Heap Analysis}
\goup
\label{sect:heapanalysis}

Since access permissions guard accesses to heap locations, our
inference requires information about the heap, for instance, to decide
whether two expressions may refer to the same heap location. The
analysis that approximates properties of the heap is crucial for the
effectiveness of the permission inference.  However, since the heap
analysis is not the main focus of this paper, we only sketch the main
ideas and notations of the implemented heap analysis here, which is an
extension of our earlier work \cite{FER08a}. We expect that more
sophisticated heap analyses such as shape analysis \cite{SRW02} could
be combined with our inference.

Our heap analysis computes an approximation of the concrete heap that
abstracts objects in the concrete heap to \emph{abstract nodes} in an
abstract heap. The abstraction identifies each object with the program
point where it is created. That is, it abstracts all concrete objects
created at the same program point by the same abstract \emph{summary
  node}. The heap analysis works modularly, that is, analyzes each
method separately. The initial heap for a method contains one abstract
node for each method argument. If two arguments may be aliases (that is,
their types do not exclude that they refer to the same object), they are
instead represented by one summary node.


We denote by $\aref$ the set of abstract nodes. The function
$\afunction{isSummary}:\funzione{\aref}{\{\statement{true},
  \statement{false}\}} $ yields whether a node is a summary node, that
is, whether it represents more than one concrete object. Summary nodes
allow us to handle potential aliasing as well as \statement{new}
statements  inside  loops, which may be executed a statically-unknown
number of times. 

Our permission inference uses type information for abstract nodes to
determine which monitor invariant to inhale and exhale for acquire and
release statements. The function
$\cfunction{class}:\funzione{\aref}{\className}$ yields the class of
an abstract node ($\className$ is the set of class identifiers); for
summary nodes, it returns the smallest superclass of
the classes of each concrete object represented by the summary node.

The function $\afunction{\fields}:\funzione{\aref}{\parts{\fieldset}}$
yields the set of fields of an abstract node, and the union of these
sets for a summary node; $\fieldset$ is the set of field identifiers.
A path in $\spath$ is a sequence of variable and field identifiers. As
usual, we denote the concatenation of
two paths $\statement{p}_1$ and $\statement{p}_2$ by
$\statement{p}_1.\statement{p}_2$.  

The heap analysis needs to define the semantics of expressions and
statements in order to describe their effects on the abstract heap.
The function $\afunction{\semanticanome{\expressionsemantics}} :
\funzione{(\aheap, \statement{E})}{\aref}$ evaluates an expression in
an abstract heap and yields the resulting abstract node (\aset{H} is
the set of abstract heaps). We assume a semantics of statements that
tracks how each statement modifies the abstract heap. We do not
present this semantics here, and we leave the heap modifications
implicit in the abstract semantics for the permission inference.


%As we already said, we rely on the design by contract approach, and so the abstract state after the execution of a method call is computed following its contracts\footnote{Clearly this is strictly related with the access permissions specified by these contracts. Our approach suppose that methods are not annotated, thus the heap analysis abstracts away all the part of the heap that is reachable through the parameters and the object on which the method is called}.





\goup
\section{Symbolic Permissions}
\goup
\label{sect:abstractdomain}

In this section we introduce the symbolic values that represent
occurrences of permission predicates in specifications, present the
formalization of the abstract domain and the abstract semantics, and
discuss an unsound abstraction that can improve the results of the
inference.

\goup
\subsection{Symbolic Values}
\goup

Programs may contain the permission predicates \statement{acc} and
\statement{rd} in pre- and postconditions, monitor invariants, and
loop invariants. We represent the permissions in the former three
places using symbolic values; we discuss the treatment of loop
invariants in \secref{sect:inference}. We represent the access
permissions for a path \statement{p.f} specified by (1)~the
precondition and postcondition of a method \statement{m} in class
\statement{C} by \presv{C}{m}{p.f} and \postsv{C}{m}{p.f},
respectively, and (2)~the monitor invariant of class \statement{C} by
\misv{C}{p.f}. Since there could be many (possible infinite) paths to
reach a location, our semantics always considers a shortest path.
We denote the set of all symbolic values by $\asymbolicvalues$.


\goup
\subsection{Abstract Domain}
\goup 

The abstract domain $\asvarvalue$ represents the \emph{symbolic access
  permission} for a single abstract location, that is, field of an
abstract node. We represent the symbolic permission as the
\emph{summation} of symbolic values $\statement{s}_i$ multiplied by
integer coefficients $\cel{a_i}$ (to represent how many times we have
inhaled or exhaled a permission) and an integer constant \cel{c} (to
represent integer permissions, for instance, for the 100\% permission
that is inhaled when an object is created).  Formally,
$\asvarvalue=\{\sum_\cel{i} \cel{a_i} * \statement{s}_i + \cel{c}
\mbox{\ where\ } \cel{a_i}, \cel{c} \in \integernumbers,
\statement{s}_i \in \asymbolicvalues \}$.

\begin{figure}
\goup
\goup
\goup
\scriptsize
$
\begin{array}{l}
(\sum_j a^1_j * \ael{s}_j + c_1) \sqcup_{\asvarvalue} (\sum_j a^2_j * \ael{s}_j + c_2) = (\sum_j \min(a^1_j, a^2_j) * \ael{s}_j + \min(c_1, c_2))\\
(\sum_j a^1_j * \ael{s}_j + c_1) \sqcap_{\asvarvalue} (\sum_j a^2_j * \ael{s}_j + c_2) = (\sum_j \max(a^1_j, a^2_j) * \ael{s}_j + \max(c_1, c_2))\\
(\sum_j a^1_j * \ael{s}_j + c_1) \leq_{\asvarvalue} (\sum_j a^2_j * \ael{s}_j + c_2) = \true \Leftrightarrow c_1 \geq c_2 \land \forall j : a^1_j \geq a^2_j\\
\end{array}
$
\normalsize
\goup
\caption{Lattice operators on $\asvarvalue$.}
\label{fig:latticeoperators}
\goup
\goup
\goup
\end{figure}

On these summations we define a lattice structure.
\figref{fig:latticeoperators} formalizes lattice operators. In order
to be sound, we compute at each program point the access permissions
the current thread \emph{surely} has, that is, it has in all possible
executions. For this reason, the upper bound---which is used in the
abstract semantics for joins in the control flow---of two symbolic
permissions $\ael{l}_1$ and $\ael{l}_2$ is the \emph{minimum} of
$\ael{l}_1$ and $\ael{l}_2$. Since symbolic values represent
non-negative values, the minimum of two symbolic permissions is
computed using the minimum of the corresponding coefficients and of
the integer constant \cel{c}. Conversely, the lower bound is the
\emph{maximum} of $\ael{l}_1$ and $\ael{l}_2$, which is computed
analogously. Symbolic permission $\ael{l}_1$ is less or equal $\ael{l}_2$ iff
$\ael{l}_1$ represents \emph{greater} or equal permissions than
$\ael{l}_2$, that is, each coefficient and the constant in $\ael{l}_1$
is greater or equal than the corresponding coefficient and the
constant in $\ael{l}_2$; this definition is in line with defining the
upper bound as the minimum.  The bottom element is any value greater
than 100 or less than 0, that is, any invalid value for a
permission. The top element is the value 0. Then the lattice can be
defined by $\langle \asvarvalue, \leq_{\asvarvalue}, 101, 0,
\sqcup_{\asvarvalue}, \sqcap_{\asvarvalue} \rangle$. Note that this
domain does not track disjunctive information like $\cel{b}
\Rightarrow \cel{acc}(\statement{x.f})$, but it can be used inside
other generic domains to obtain precise disjunctive information
\cite{MR05}.

In the domain we defined, we have a finite number of symbolic values,
but the integer coefficients could decrease indefinitely. Therefore,
we need a widening operator to ensure the termination of the analysis.
Our widening operator abstracts the symbolic value to top if it is
decreasing. This definition reflects that if a loop exhales
permissions in each iteration, we have to assume that no permission is
left when the loop terminates because we do not know statically how
many times the loop body will be executed. However, we did not have to apply 
widening in any of the examples we analyzed.


%% This result seems reasonable: if the permission decreases
%% each time we execute the body of a loop, it would mean that we are
%% exhaling something at each iteration, and statically we may not know
%% precisely how many times we are going to iterate the loop. On the
%% other hand, it is possible to write programs that exhale something at
%% each iteration of a loop, given the hypothesis that we control the
%% number of loop iterations, and we may not track precisely this
%% information because of abstraction. Nevertheless, we think and we find
%% out that in practice this approach is almost never used.

Our abstract domain tracks the access permissions at a given program
point for each field of each abstract node. Therefore, its state is
represented by a function that maps abstract locations to symbolic
permissions: $\apermission : \funzione{(\aref \times \fieldset)}{\asvarvalue}$,
where $\apermission$ is the set representing our abstract domain. The
lattice operators are defined as the functional extensions of the
lattice operators of $\asvarvalue$.


\goup
\subsection{Abstract Semantics}
\goup
\label{subsect:abstractsemantics}



\begin{figure}[b!]
\goup
\goup
\goup
\scriptsize
$
\begin{array}{l}
\afunction{\reachable} : \funzione{(\aref \times \aheap \times \parts{\aref} \times \spath)}{\parts{\aref \times \fieldset \times \spath}}\\
\areachable{\ael{r}}{\ael{h}}{\aset{R}}{\statement{p}} = \{ (\ael{r}_1, \statement{f}, \statement{p}) : (\ael{r}_1, \statement{f}) \in \aonestepreachable{\ael{h}}{\statement{p}} \land \ael{r}_1 \notin \aset{R} \} \cup\\
\hspace{10pt} \cup \{ (\ael{r}_2, \statement{f_1}, \cel{p}_1) \in \areachable{\ael{r}_1}{\ael{h}}{\aset{R} \cup \downarrow_1(\aonestepreachable{\ael{h}}{\statement{p}})}{\cel{p}.\statement{f}} : (\ael{r}_1, \statement{f}) \in \aonestepreachable{\ael{h}}{\statement{p}} \}\\
\\
\afunction{\onestepreachable} : \funzione{(\aheap \times \spath)}{\parts{\aref \times \fieldset}}\\
\aonestepreachable{\ael{h}}{\statement{p}} = \{ (\ael{r}, \cel{f}) : \aexpressionsemantics{\ael{h}}{\statement{p}}=\ael{r} \land \statement{f} \in \afields{\ael{r}} \}\\
\\
%
\afunction{\replace} :\funzione{(\parts{\aref \times \fieldset \times \spath} \times \asymbolicvalues)}{\parts{\aref \times \fieldset \times \asymbolicvalues}}\\
\areplace{\{(\ael{r}_1, \statement{f}_1, \statement{p}_1), \cdots, (\ael{r}_i, \statement{f}_i, \statement{p}_i)\}}{\ael{s}} = \{(\ael{r}_1, \statement{f}_1, \ael{s}_1), \cdots, (\ael{r}_i, \statement{f}_i, \ael{s}_i) \} : \forall \cel{j} \in [1..\cel{i}] : \\
\hspace{100pt} \ael{s}_j =
\left\{
\begin{array}{ll}
\misv{\statement{c}}{\statement{p_j}.\statement{f_j}} & \textrm{if } \ael{s}=\misv{\statement{c}}{\statement{p}}\\
\presv{\statement{c}}{\statement{m}}{\statement{p_j}.\statement{f_j}} & \textrm{if } \ael{s}=\presv{\statement{c}}{\statement{m}}{\statement{p}}\\
\postsv{\statement{c}}{\statement{m}}{\statement{p_j}.\statement{f_j}} & \textrm{if } \ael{s}=\postsv{\statement{c}}{\statement{m}}{\statement{p}}\\
%\apsv{\statement{c}}{\statement{pred}}{\statement{p_j}.\statement{f_j}} & \textrm{if } \ael{s}=\presv{\statement{c}}{\statement{pred}}{\statement{p}}\\
\end{array}
\right.
\}
\\
\\
%
\afunction{\inhalesingle} :\funzione{(\apermission \times \aref \times \fieldset \times \asymbolicvalues)}{\apermission}\\
\ainhalesingle{\ael{\sigma}}{\ael{r}}{\statement{f}}{\ael{s}}=
\left\{
\begin{array}{ll}
\ael{\sigma}[(\ael{r}, \statement{f}) \mapsto \ael{\sigma}(\ael{r}, \statement{f})+1*\ael{s}] &\textrm{ if } \aisSummary{\ael{r}}=\false\\
\ael{\sigma} & \textrm{ otherwise}
\end{array}
\right.
\\
\afunction{\inhale} :\funzione{(\apermission \times \parts{\aref \times \fieldset \times \asymbolicvalues})}{\apermission}\\
\ainhale{\ael{\sigma}}{\{(\ael{r}_1, \statement{f_1}, \ael{s}_1), \cdots, (\ael{r}_i, \statement{f_i}, \ael{s}_i)\}} = \ael{\sigma}_i :\\
\hspace{100pt} \exists \ael{\sigma}_0, \cdots, \ael{\sigma}_i \in \apermission : \ael{\sigma}_0=\ael{\sigma} \land \forall \cel{j} \in [1..\cel{i}] : \ael{\sigma}_j=\ainhalesingle{\ael{\sigma}_{j-1}}{\ael{r}_j}{\statement{f_j}}{\ael{s}_j}\\
\\
\afunction{\exhalesingle} :\funzione{(\apermission \times \aref \times \fieldset \times \asymbolicvalues)}{\apermission}\\
\aexhalesingle{\ael{\sigma}}{\ael{r}}{\statement{f}}{\ael{s}}= \ael{\sigma}[(\ael{r}, \statement{f}) \mapsto \ael{\sigma}(\ael{r}, \statement{f})-1*\ael{s}]\\
\\
\afunction{\exhale} :\funzione{(\apermission \times \parts{\aref \times \fieldset \times \asymbolicvalues})}{\apermission}\\
\aexhale{\ael{\sigma}}{\{(\ael{r}_1, \statement{f_1}, \ael{s}_1), \cdots, (\ael{r}_i, \statement{f_i}, \ael{s}_i)\}} = \ael{\sigma}_i : \\
\hspace{100pt} \exists \ael{\sigma}_0, \cdots, \ael{\sigma}_i \in \apermission : \ael{\sigma}_0=\ael{\sigma} \land \forall \cel{j} \in [1..\cel{i}] : \ael{\sigma}_j=\aexhalesingle{\ael{\sigma}_{j-1}}{\ael{r}_j}{\statement{f_j}}{\ael{s}_j}
\end{array}
$
\normalsize
\goup
\caption{Helper functions for the abstract semantics.  The prefix
  operator $\downarrow_1$ denotes the projection of a pair on its
  first component; it is lifted to sets of pairs.  }
\label{fig:semanticsfunction}
\goup
\goup
\end{figure}

The abstract semantics formalizes the effect of statements on symbolic
permissions. It uses the helper functions in \figref{fig:semanticsfunction}. 
$\areachable{\ael{r}}{\ael{h}}{\aset{R}}{\statement{p}}$ yields the
set of abstract locations that can be reached from an abstract node
$\ael{r}$ in a heap $\ael{h}$, without traversing the abstract nodes
in $\aset{R}$, and for each reachable abstract location a path through
which it can be reached; this path is an extension of path
$\statement{p}$, through which node $\ael{r}$ is reachable from some
starting point. The set $\aset{R}$ is used to discard alternative paths 
to the same abstract location.
%
Note that the definition of $\afunction{\reachable}$ is recursive. For
each recursive application, we use function
$\afunction{\onestepreachable}$ to add all abstract locations that are
reachable in one step, that is, by accessing a field of $\ael{r}$.
The recursion is well-founded since the heap domain contains a finite
number of abstract nodes, and the set $\aset{R}$ grows in each
recursive application.

Function $\afunction{\reachable}$ is used to extract all the abstract
locations for which we \emph{potentially} inhale or exhale
permissions, together with a shortest path through which these
permissions could be inhaled or exhaled.  Function
$\afunction{\replace}$ uses these abstract locations and paths to
construct a symbolic value for each of them.  Its last argument
determines what kind of symbolic value we want to obtain.  

%With this process we consider only the shortest path that can be used to access an abstract location, but there could be many other (possible infinite) ways of accessing such location, e.g., when dealing with recursive data structures.

Finally, we define two functions $\afunction{\inhalesingle}$ and
$\afunction{\exhalesingle}$ to inhale and exhale permissions,
respectively. They map a state of the abstract domain to another
state. The permissions are determined by pairs of abstract locations
and symbolic values. The functions $\afunction{\inhale}$ and
$\afunction{\exhale}$ lift $\afunction{\inhalesingle}$ and
$\afunction{\exhalesingle}$ to sets of pairs.
Note that we inhale a permission iff the abstract node of its location
is not a summary node; otherwise we would potentially inhale one
permission for multiple concrete locations, which is unsound.

\begin{figure}
\goup
\goup
\goup
\scriptsize
$
\begin{array}{l}
\afunction{\semanticanome{\statementsemantics}} : \funzione{(\statements, \apermission, \aheap)}{\apermission}\\

\astatementsemantics{\statement{x:=E}}{\ael{\sigma}}{\ael{h}} = \ael{\sigma}\\
\astatementsemantics{\statement{x.f:=E}}{\ael{\sigma}}{\ael{h}} = \ael{\sigma}\\

\astatementsemantics{\statement{x := new\ T}}{\ael{\sigma}}{\ael{h}} = \ael{\sigma}[\ael{r} \mapsto 100 : (\ael{r}, \statement{p}) \in \aonestepreachable{\ael{h'}}{\statement{x}}]\\
\hspace{100pt} \textrm{ where } \ael{h'} \textrm{ is the abstract heap obtained after } \statement{x := new\ T}\\

\astatementsemantics{\statement{x.m()}}{\ael{\sigma}}{\ael{h}} = \ael{\sigma}_2 : \ael{\sigma}_1=\aexhale{\sigma}{\areplace{\areachable{\aexpressionsemantics{\ael{h}}{\statement{x}}}{\ael{h}}{\emptyset}{\statement{this}}}{\presv{\class{\aexpressionsemantics{\ael{h}}{\statement{x}}}}{\statement{m}}{\emptyset}}} \land\\
\hspace{70pt} \ael{\sigma}_2=\ainhale{\ael{\sigma}_1}{\areplace{\areachable{\aexpressionsemantics{\ael{h}}{\statement{x}}}{\ael{h}}{\emptyset}{\statement{this}}}{\postsv{\class{\aexpressionsemantics{\ael{h}}{\statement{x}}}}{\statement{m}}{\emptyset}}} \\

\astatementsemantics{\statement{acquire\ x}}{\ael{\sigma}}{\ael{h}} = \ainhale{\sigma}{\areplace{\areachable{\aexpressionsemantics{\ael{h}}{\statement{x}}}{\ael{h}}{\emptyset}{\statement{this}}}{\misv{\class{\aexpressionsemantics{\ael{h}}{\statement{x}}}}{\emptyset}}}\\
\astatementsemantics{\statement{release\ x}}{\ael{\sigma}}{\ael{h}} = \aexhale{\sigma}{\areplace{\areachable{\aexpressionsemantics{\ael{h}}{\statement{x}}}{\ael{h}}{\emptyset}{\statement{this}}}{\misv{\class{\aexpressionsemantics{\ael{h}}{\statement{x}}}}{\emptyset}}}\\

\astatementsemantics{\statement{t:=fork\ x.m()}}{\ael{\sigma}}{\ael{h}} = \aexhale{\sigma}{\areplace{\areachable{\aexpressionsemantics{\ael{h}}{\statement{x}}}{\ael{h}}{\emptyset}{\statement{this}}}{\presv{\class{\aexpressionsemantics{\ael{h}}{\statement{x}}}}{\statement{m}}{\emptyset}}}\\
\astatementsemantics{\statement{join\ t}}{\ael{\sigma}}{\ael{h}} = \ainhale{\sigma}{\areplace{\areachable{\aexpressionsemantics{\ael{h}}{\statement{x}}}{\ael{h}}{\emptyset}{\statement{this}}}{\postsv{\statement{C}}{\statement{m}}{\emptyset}}} : \cthreadmethod{\statement{t}}=\statement{C.m}\\

\astatementsemantics{\statement{share\ x}}{\ael{\sigma}}{\ael{h}} = \aexhale{\sigma}{\areplace{\areachable{\aexpressionsemantics{\ael{h}}{\statement{x}}}{\ael{h}}{\emptyset}{\statement{this}}}{\misv{\class{\ael{h}(\statement{x})}}{\emptyset}}}\\

%\astatementsemantics{\statement{x.f := new\ T}}{\ael{\sigma}}{\ael{h}} = \ael{\sigma}[\ael{r} \mapsto 100 : (\ael{r}, \statement{p}) \in \aonestepreachable{\aexpressionsemantics{\ael{h}}{\statement{new\ T}}}{\ael{h}}{\statement{x.f}}]\\


%\astatementsemantics{\statement{new\ T}}{\ael{\sigma}}{\ael{h}} = \ael{\sigma}[\ael{r} \mapsto 100 : (\ael{r}, \statement{p}) \in \aonestepreachable{\aexpressionsemantics{\ael{h}}{\statement{new\ T}}}{\ael{h}}{\emptyset}]\\
%\astatementsemantics{\statement{free\ x}}{\ael{\sigma}}{\ael{h}} = \ael{\sigma}[\ael{r} \mapsto 0 : (\ael{r}, \statement{p}) \in \aonestepreachable{\aexpressionsemantics{\ael{h}}{\statement{x}}}{\ael{h}}]\\

%\astatementsemantics{\statement{fold\ x.p}}{\ael{\sigma}}{\ael{h}} = \aexhale{\sigma}{\areplace{\areachable{\aexpressionsemantics{\ael{h}}{\statement{x}}}{\ael{h}}{\emptyset}{\statement{this}}}{\apsv{\class{\aexpressionsemantics{\ael{h}}{\statement{x}}}}{\statement{p}}{\emptyset}}}\\
%\astatementsemantics{\statement{unfold\ x.p}}{\ael{\sigma}}{\ael{h}} = \ainhale{\sigma}{\areplace{\areachable{\aexpressionsemantics{\ael{h}}{\statement{x}}}{\ael{h}}{\emptyset}{\statement{this}}}{\apsv{\class{\aexpressionsemantics{\ael{h}}{\statement{x}}}}{\statement{p}}{\emptyset}}}\\

%\astatementsemantics{\statement{unshare\ x}}{\ael{\sigma}}{\ael{h}} = \ainhale{\sigma}{\areplace{\areachable{\aexpressionsemantics{\ael{h}}{\statement{x}}}{\ael{h}}{\emptyset}{\statement{this}}}{\misv{\class{\aexpressionsemantics{\ael{h}}{\statement{x}}}}{\emptyset}}}\\


\end{array}
$
\normalsize
\goup
\caption{The definition of the abstract semantics.  The function
  $\cfunction{\threadmethod}$ yields the method with which a given
  thread was forked.}
\label{fig:abstractsemantics}
\goup
\goup
\end{figure}

The abstract semantics of statements (\figref{fig:abstractsemantics})
maps a statement, a state of the abstract domain, and a heap to
another state. It reflects the permission transfer described in
\secref{sect:language}.  For instance, acquiring a monitor inhales all
the symbolic permissions that its invariant could specify. These are
permissions for all abstract locations reachable from the object whose
monitor is being acquired. To determine these abstract locations we apply 
$\afunction{\replace}$ to the result of $\afunction{\reachable}$.


\goup
\goup
\subsubsection{Running example.}  

In method
\statement{Inc} of class \statement{W1}, we obtain that between the
\statement{acquire} and the \statement{release} statements (lines 7
and 8), the current thread has the symbolic access permission
$1*$\presv{W1}{Inc}{this.c.f}+1*\misv{Cell}{this.f}
for each field \statement{f} of \statement{Cell} (that is, \statement{x}, \statement{c1}, or \statement{c2}). At the end of the method, it has only
$1*$\presv{W1}{Inc}{this.c.f}
since we released the monitor of \statement{c}. The permissions for
class \statement{W2} are analogous.


Before the fork in method \statement{main} of class
\statement{OwickiGries}, the current thread has
$-1*$\misv{Cell}{c.f}$+100$ for all the fields \statement{f} of class
\statement{Cell}. The negated permissions from the monitor invariant
stem from exhaling the monitor invariant when sharing \statement{c};
the constant 100 is inhaled when \statement{c} is created.  When
forking the two threads (line 36), we exhale the precondition of the
forked methods, obtaining
$-1*$\misv{Cell}{c.f}$-1*$\presv{W1}{Inc}{c.f}$-1*$\presv{W2}{Inc}{c.f}$+100$.
When joining the forked threads, we inhale the postconditions of the
forked methods, and when acquiring \statement{c}'s monitor (line 39), we 
inhale the monitor invariant of class \statement{Cell}, regaining what we
lost when sharing it. Therefore, at line~41, the current thread has:
$-1*$\presv{W1}{Inc}{c.f}$+1*$\postsv{W1}{Inc}{c.f}$-1*$\presv{W2}{Inc}{c.f}$+1*$\postsv{W2}{Inc}{c.f}$+100$ for
each field \statement{f} of \statement{Cell}.

\goup
\subsection{Unsoundness to Improve the Inference}
\goup

The analysis we have described so far is sound, but sometimes too
coarse in its treatment of summary nodes. Even with a more precise
heap analysis, the inference  becomes more practical when it uses
two unsound approximations, which we describe next.

First, as we explained earlier, a sound analysis must not inhale
permissions on summary nodes because this might result in the
duplication of permissions. Removing this restriction improves
especially the treatment of recursive data structures, which 
are usually abstracted to summary nodes.

Second, our analysis conservatively assumes maximum aliasing in the
input state of a method, that is, arguments or fields whose types do
not rule out aliasing are represented by summary nodes. Following
Clousot \cite{LF10}, we suggest to assume that aliasing does not occur
in the input state. This unsound assumption is useful when methods
take several parameters of the same type as well as when a parameter
is a recursive data structure.

These unsound approximations may lead to permission annotations that
are insufficient to verify a program. For instance, unsoundly inhaling
on a summary node representing \statement{x} and \statement{y}
 might provide enough permissions to permit the access \statement{x.f}
even if in the real execution, there is only a permission for \statement{y.f}.
However, in our experiments (see \secref{sect:experimentalresults}), 
the unsound approximations helped inferring complete
annotations, without compromising their quality.


\goup
\section{Annotation Inference}
\goup
\label{sect:inference}

In this section, we explain how we infer constraints from the
information tracked by the abstract domain, and how these constraints
are solved.

\goup
\subsection{Inferring Constraints}
\goup


Through the abstract semantics, we obtain a symbolic permission for
each abstract location at each program point. When we access an
abstract location, we impose some constraints on these symbolic
permissions. In particular, given a symbolic permission $\ael{l}$, we
have to ensure that (1)~$\ael{l} \geq \epsilon$ when we read,
(2)~$\ael{l} = 100$ when we write, (3)~$\ael{l} \leq 100$ after we
inhale, and (4)~$\ael{l} \geq 0$ after we exhale.
To ensure that specifications are self-framing (see \secref{fig:language}), we generate
constraint~(1) also for field accesses within preconditions, 
postconditions, and monitor invariants.
%(v) $\ael{l} \geq \misv{\statement{c}}{\statement{p}}$ when we unshare an object, where \statement{c} is the class it is instance of, $\ael{l}$ is one of the abstract locations that are reachable from the unshared object, and \statement{p} is the path used to access it; 
%(vi) $\ael{l} == 100$ before \statement{free\ x} where $\ael{l}$ is the permission level of an abstract location that is in $\downarrow_1(\aonestepreachable{\aexpressionsemantics{\ael{h}}{\statement{x}}}{\ael{h}}{\statement{x}})$ and $\ael{h}$ is the state of the abstract heap at that point.
Moreover, all symbolic values must represent valid permissions:
$\forall \ael{s} \in \asymbolicvalues : 0
\leq \ael{s} \leq 100$.
We use $\ael{l} \geq \epsilon$ instead of $\ael{l} > 0$ in constraint (1) to enforce that the constraint solver does not produce non-zero permissions smaller than $\epsilon$, which is the smallest permissions that can be expressed in Chalice.


We use the symbolic value $\epsilon$ to represent an infinitesimal
permission. The concrete value of $\epsilon$ must be a read permission
($0< \epsilon$). Recall that $\epsilon$ is an infinitesimal
permission, which acts like a counting permission, that is, a thread
can have several $\epsilon$-permissions and also negative
$\epsilon$-permissions (for instance, by exhaling an
$\epsilon$-permissions from a full permission). To reflect these
properties, we require $n*\epsilon < 0.5$, where $n$ is the maximal
coefficient for $\epsilon$ in all symbolic permissions. We interpret values in the open interval $(0;0.5)$ as
a positive number of $\epsilon$'s and values in $(0.5;1)$ as 1 plus a
negative number of $\epsilon$'s.


Finally, in order to infer postconditions, we introduce additional
constraints for the exit states of the analysis.  For each field of a
non-summary node reachable through a path \statement{p}, we determine
the upper bound $\ael{l}$ of the symbolic permissions for all possible exit
states of the method.  We then require
$\postsv{C}{m}{p}=\ael{l}$ where \statement{m} and \statement{C} are
the method and the class we are analyzing.





\goup
\goup
\subsubsection{Running Example.} 

\figref{fig:runningexampleconstraints} reports some of the
constraints for the example from \figref{fig:runningexample}.  We have
already discussed the results of its abstract semantics in the
previous section. For each constraint, we report the code line that
induced the constraints. As before, \statement{f} stands for any field
of class \statement{Cell} (\statement{x}, \statement{c1}, or
\statement{c2}).

\begin{figure}
\goup
\goup
\goup
\scriptsize
\begin{center}
\begin{tabular}{|l|c|}
\hline
\textbf{Constraint} & \textbf{Line}\\
\hline
1*\presv{W1}{Inc}{c.c1} $\geq\epsilon$ & \statement{4}\\

1*\presv{W1}{Inc}{c.x} + 1*\misv{Cell}{x}=100 & \statement{7}\\

1*\presv{W1}{Inc}{c.c1} + 1*\misv{Cell}{c1}=100  & \statement{8}\\

1*\postsv{W1}{Inc}{c.f} = 1*\presv{W1}{Inc}{c.f} & \statement{10}\\

%\hline

%1*\presv{W2}{Inc}{c.c2} $>0$ & \statement{19}\\

%1*\presv{W2}{Inc}{c.x} + 1*\misv{Cell}{x}=100 & \statement{22}\\

%1*\presv{W2}{Inc}{c.c2} + 1*\misv{Cell}{c2}=100 & \statement{23}\\

%1*\postsv{W2}{Inc}{c.f} = 1*\presv{W2}{Inc}{c.f} & \statement{25}\\

\hline

1*\misv{Cell}{f} $\geq\epsilon$ & \statement{14}\\

\hline

100 - 1*\misv{Cell}{f} - 1*\presv{W1}{Inc}{c.f} $\geq$ 1*\presv{W2}{Inc}{c.f}& \statement{36}\\

\hline
\end{tabular}
\normalsize
\goup
\caption{Some constraints for the running example.\label{fig:runningexampleconstraints}
}
\end{center}
\goup
\goup
\goup
\goup
\end{figure}

\lstset{basicstyle=\sffamily\normalsize}
The first four constraints are introduced for method \statement{Inc}
of class \statement{W1}. The constraints for class \statement{W2} are
analogous (with \statement{c2} instead of \statement{c1}).
The constraint for line~4 is introduced because \statement{Inc}'s
postcondition reads \statement{c.c1}; in the exit state of the method,
the only permission for \statement{c.c1} is the one specified in the
precondition since we already released the monitor of \statement{c}.
The identical constraint is introduced for the field read
\lstinline{old(c.c1)}, which reads \statement{c.c1}'s pre-state value.

The field writes to \statement{c.x} and \statement{c.c1} (lines 7
and 8) require that the thread has 100\% permission
for the corresponding abstract locations. Therefore, using the
information inferred by the abstract semantics, we introduce a
constraint that the access permission for these locations specified in
\statement{Inc}'s precondition plus the permission specified in
\statement{Cell}'s monitor invariant is equal to 100.

Finally, by the abstract semantics the permissions in the exit state
of \statement{Inc} are exactly those specified in the precondition.
So we enforce that the precondition and the postcondition specify the
same permissions for each field \statement{f} of \statement{c}.
%, in order not to lose any permissions.

The class invariant of class \statement{Cell} (line 14) reads all
fields of the class and, thus, requires read permission for them.


Several constraints are produced for \statement{main} of class
\statement{OwickiGries}. We discuss the one for the second fork (line
36), which enforces that \statement{main} has sufficient
permissions for the fields of \statement{c} in order to exhale the
precondition of the forked method \statement{Inc}. By the heap
analysis, we know that \statement{c} is fresh in method
\statement{main}. So the permissions held before the second fork for
any field \statement{f} of \statement{c} are 100 (from the creation of
\statement{c}) minus what is specified in \statement{Cell}'s monitor
invariant (from sharing \statement{c}) minus what is specified in
\statement{W1.Inc}'s precondition (from the first fork). This
permission must be greater than or equal to what is specified in
\statement{W2.Inc}'s precondition in order to enable exhaling the
permissions in the second fork.




\goup
\subsection{Resolution of the Constraints}
\goup

We solve the inferred system of constraints using linear programming
\cite{DAN63}. We define an objective function that lets us infer the
\emph{minimal} access permissions that satisfy the constraints imposed
by the program. Maximizing the permission we would often obtain full
permission for each reachable locations, even if the location is never
accessed. Such a solution complicates subsequent verification, for
instance, by providing weaker framing.

Through the objective function we also express \emph{priorities} where
to put annotations when several solutions are possible, for instance,
in the method specification or in the monitor invariant. To do that,
we multiply each symbolic value in the objective function with a
factor. A bigger factor expresses a lower priority for that symbolic
value, since we minimize the objective function.

Solving the linear programming system determines whether the
system is feasible, that is, whether there are numerical values (real
numbers) for all symbolic values that satisfy the constraint.  An
infeasible system may occur because of approximation, for instance, if
we soundly abstain from inhaling on summary nodes, the constraint for
a subsequent field access might not be satisfiable.

If the system is feasible, we use the solution to compute the
permission predicates for pre- and postconditions as well as monitor
invariants. The integer part of each numerical value is turned into
the percentage of a fractional permission, whereas the mantissa is
turned into (positive or negative) counting permissions by dividing it
by the solution for the symbolic value $\epsilon$.

In addition, we compute the permission predicates for \emph{loop
  invariants}. We first determine the upper bound of the symbolic
permissions in the abstract entry and exit state of each loop and then
replace the symbolic values in these symbolic permissions by the
numerical values from the linear programming system, thereby obtaining
the permissions held both before and after a loop iteration. The then
use the $\afunction{\reachable}$ analysis from
\figref{fig:semanticsfunction} to determine which of these permissions
can be added to the loop invariant.


\goup
\goup
\subsubsection{Running Example.} 

The linear programming system for our example
(\figref{fig:runningexampleconstraints}) is feasible. We obtain
different solutions, depending on priorities encoded in the objective
function. If we give priority to monitor invariants, we get the
smallest possible solution for the pre- and postconditions that
satisfy the constraints. Let's assume that the numerical value for
$\epsilon$ is $0.1$. Then for \statement{W1}'s \statement{Inc} method we obtain
$0.1$ for \statement{c.c1} and $0$ for all other fields,
which are the smallest possible values that satisfy the constraints
(especially the first) in \figref{fig:runningexampleconstraints}.
This solution results in \statement{rd(c.c1)} for the pre- and
postcondition, and analogous for \statement{W2.Inc}.

Since the sum of the permissions for any \statement{c.f} in the
precondition and in the monitor invariant has to be 100, we obtain
$99.9$ (that is, $100-\epsilon$) for \statement{c.c1} and
\statement{c.c2}, and $100$ for \statement{c.x}. This solution cannot
be expressed in Chalice, which does not have syntax for a negative
number of $\epsilon$'s. However, we could easily add a constraint that
for each symbolic value, the mantissa of the numerical value in the
solution must be in $[0;0.5)$ and, thus, translate into a
  non-negative number of $\epsilon$'s. With this additional constraint,
we obtain the pre- and postcondition \statement{acc(c.c1,1)}, and the
monitor invariant \lstinline{acc(x) && acc(c1,99) && acc(c2,99)}.


%The same considerations apply to method \statement{Inc} of class \statement{W2} replacing \statement{c1} with \statement{c2}. So we obtain the following annotation: 
%
%\setlength{\columnseprule}{0.5pt}
%\setlength{\columnsep}{20pt}
%\begin{multicols}{3}
%\begin{lstlisting}[numbers=none]
%class W1 {
%  method Inc()
%  requires rd(c1)
%  ensures rd(c1) {...}
%}
%\end{lstlisting}
%\columnbreak
%\begin{lstlisting}[numbers=none]
%class W2 {
%  method Inc()
%  requires rd(c2)
%  ensures rd(c2) {...}
%}
%\end{lstlisting}
%\columnbreak
%\begin{lstlisting}[numbers=none]
%class Cell {
% invariant acc(x) && 
%   acc(c1, 100-$\epsilon$) && 
%   acc(c2, 100-$\epsilon$)
%}
%\end{lstlisting}
%\end{multicols}
%Note that $100-\epsilon$ is not valid in Chalice, since it is not possible to specify a negative amount of epsilons. On the other hand, Chalice implementation encodes negative amounts of epsilons, and we think that these could be easily supported in future releases. Anyway, the solution we proposed is effectively the one that maximizes the annotation on monitor invariants, but we may impose more constraints in our system to avoid that negative amount of epsilons appears in the annotation.


Putting the priority on method specification results in 99\%
permissions for \statement{c.c1} in the pre- and postconditions of
\statement{W1.Inc} (analogous for \statement{c.c2} for
\statement{W2.Inc}) and 1\% each for \statement{c1} and \statement{c2}
in \statement{Cell}'s monitor invariant to satisfy the fifth constraint
in \figref{fig:runningexampleconstraints}. By the second, third, and
last constraint, the
preconditions of the \statement{Inc} methods must not contain any
permission for \statement{c.x}. So by the second constraint, the
monitor invariant contains the full permission for \statement{x}.
Assigning equal priority to method specifications and monitor
invariants splits the permissions for \statement{c1} and
\statement{c2} 50:50.  All three annotations reflect the need to split
the permissions as discussed in \secref{sect:runningexample}, and they
allow one to verify the example in Chalice.






%In the second scenario, we maximize the annotation on contracts. In this context, we can assign at most $100\%-\epsilon$ permissions on the pre and post conditions of methods \statement{Inc} in class \statement{W1} (on \statement{c.c1}) and in class \statement{W2} (on \statement{c.c2}), since we need at least a read permission on these fields in the monitor invariant of class \statement{Cell}. On the other hand, the constraints induced by the analysis of method \statement{main} in class \statement{OwickiGries} does not allow to infer 100\% on \statement{c.x} in \statement{W1.Inc} and \statement{W2.Inc}. In fact, after the two forks, we have that the sum of the preconditions of method \statement{Inc} in classes \statement{W1} and \statement{W2} and of the class invariant on that field has to be less or equal than 100\%. Then the only way we have to obtain 100\% permission is to discharge all this permission on the monitor invariants of class \statement{Cell}. 
%Then we obtain the following annotation: 
%
%\setlength{\columnseprule}{0.5pt}
%\setlength{\columnsep}{20pt}
%\begin{multicols}{3}
%\begin{lstlisting}[numbers=none]
%class W1 {
%  method Inc()
%  requires acc(c1, 100-$\epsilon$)
%  ensures acc(c1, 100-$\epsilon$) {...}
%}
%\end{lstlisting}
%\columnbreak
%\begin{lstlisting}[numbers=none]
%class W2 {
%  method Inc()
%  requires acc(c2, 100-$\epsilon$)
%  ensures acc(c2, 100-$\epsilon$) {...}
%}
%\end{lstlisting}
%\columnbreak
%\begin{lstlisting}[numbers=none]
%class Cell {
% invariant acc(x) && 
%           rd(c1) && 
%           rd(c2)
%}
%\end{lstlisting}
%\end{multicols}

%Finally, we could assign the same priority to monitor invariants and pre and post conditions. In this context, we obtain the 50\% of the permission for \statement{c1} and \statement{c2} in \statement{Cell}'s monitor invariants, and the 50\% in the pre and post conditions on methods \statement{Inc} in classes \statement{W1} and \statement{W2} respectively. We are still obliged to discharge 100\% permission of \statement{x} on \statement{Cell}'s monitor invariants.


%\subsection{Modularity}
%Our approach is modular since we can analyze and infer constraints on each method separately. On the other hand, if we analyze many methods together, the inferred annotation will be valid on the complete system, while if we analyze each method separately the inferred annotation may be not effective for the overall system. For instance, in the running example we could infer constraints considering only method \statement{Inc} in classes \statement{W1} and \statement{W2}. In this context it would be valid to infer 100\% permission on \statement{x} as pre conditions, but this annotation is not valid when considering method \statement{main} in class \statement{OwickiGries}, since we exhale both the preconditions when we fork \statement{t1} and \statement{t2}.
%
%In general this means that if we infer constraints on each method in isolation we could obtain annotation that is not correct for the overall program. On the other hand, the abstract semantics and the constraints inferred on a method is completely modular. The additional value of considering more methods is to put together a wider set of constraints, inferring annotation that is valid in a wider context.

\goup
\section{Experimental Results}
\goup
\label{sect:experimentalresults}

We have developed a prototype implementation of our inference in
\Sample. We execute the analysis on an Intel Code 2 Quad CPU 2.83 GHz
with 4 GB of RAM, running Windows 7, and the Java SE Runtime
Environment 1.6.0\_16-b01. Table~\ref{table:experimental_results}
summarizes the experimental results when we apply the analysis to some
case studies from the Chalice tutorial \cite{LMS09} and the Chalice
distribution. Column \statement{Program} reports the program we
analyze, and \texttt{\#C} the number of contracts dealing with access
permissions in the original program that we want to infer. We analyze
each program with the sound analysis (column \texttt{Sound}), with the
unsound entry state (\texttt{Uns.entry}), with the unsound inhaling on
summary nodes (\texttt{Uns.inhale}), and with both unsound
approximations (\texttt{Unsound}). For each of these analyses, we
report the number of contracts we inferred correctly  (\texttt{\#C}) and
the time of the analysis in milliseconds (\texttt{t}). Note that this
time includes the heap analysis, the inference of symbolic permissions, and
the time required by the linear programming library to solve the
system (in average, solving takes about 5\% of the overall time). In
the \texttt{\#C}-columns, an \textbf{X} means that the resolution of the
constraints system fails, while underlined values mean that we infer
all annotations of the original program.

\begin{table}[ht]
\goup
\goup
\goup
\begin{center}
\scriptsize
\begin{tabular}{|l||r|rr|rr|rr|rr|}																			
\hline																			
\statement{Program}	&	\texttt{\#C}	&	\multicolumn{2}{|c|}{\texttt{Sound}}			&	\multicolumn{2}{|c|}{\texttt{Uns.entry}}			&	\multicolumn{2}{|c|}{\texttt{Uns.inhale}}			&	\multicolumn{2}{|c|}{\texttt{Unsound}}			\\
	&		&	\texttt{\#C}	&	\texttt{t}	&	\texttt{\#C}	&	\texttt{t}	&	\texttt{\#C}	&	\texttt{t}	&	\texttt{\#C}	&	\texttt{t}	\\
\hline																			
\hline																			
\statement{Fig1}	&	4	&	3	&	166	&	\textbf{X}	&	173	&	3	&	169	&	\underline{4}	&	181	\\
\statement{Fig2}	&	3	&	1	&	122	&	\underline{3}	&	125	&	2	&	123	&	\underline{3}	&	124	\\
\statement{Fig3}	&	4	&	2	&	74	&	\underline{4}	&	76	&	3	&	77	&	\underline{4}	&	79	\\
\statement{Fig4}	&	4	&	\textbf{X}	&	132	&	\textbf{X}	&	131	&	3	&	131	&	\underline{4}	&	134	\\
\statement{Fig5}	&	3	&	\textbf{X}	&	220	&	\textbf{X}	&	219	&	2	&	227	&	\underline{3}	&	251	\\
\statement{Fig6}	&	14	&	7	&	140	&	\underline{14}	&	143	&	7	&	142	&	\underline{14}	&	143	\\
\statement{Fig11}	&	7	&	3	&	117	&	\underline{7}	&	122	&	\underline{7}	&	114	&	\underline{7}	&	117	\\
\statement{Fig12}	&	7	&	3	&	109	&	7	&	119	&	7	&	115	&	\underline{7}	&	118	\\
\statement{Fig13}	&	12	&	5	&	354	&	\underline{12}	&	810	&	5	&	359	&	\underline{12}	&	802	\\
\statement{AssociationList}	&	11	&	1	&	1271	&	\textbf{X}	&	2056	&	3	&	1266	&	\textbf{X}	&	2063	\\
\statement{cell-defaults}	&	25	&	\textbf{X}	&	221	&	\textbf{X}	&	234	&	14	&	222	&	\underline{25}	&	245	\\
\statement{HandOverHand}	&	25	&	6	&	1635	&	\textbf{X}	&	2752	&	8	&	1632	&	\textbf{X}	&	2788	\\
\statement{linkedlist}	&	4	&	2	&	271	&	\underline{4}	&	910	&	2	&	274	&	\underline{4}	&	918	\\
\statement{swap}	&	4	&	2	&	95	&	\underline{4}	&	110	&	2	&	110	&	\underline{4}	&	113	\\
\hline																			
\end{tabular}
\normalsize
\caption{Experimental results.}
\label{table:experimental_results}
\end{center}
\goup
\goup
\goup
\goup
\goup
\goup
\goup
\end{table}

The performance numbers indicate that our inference is fast; it
usually converges in 200 milliseconds, and in the worst case still in
less than 2.8 seconds. 

We evaluate the precision of the analysis by the number of annotations
it inferred correctly. The sound analysis was unable to infer many
contracts (around half of the annotations in average), and was unable
to infer annotations for three examples.  The unsound entry state
improves the analysis in many cases (in seven cases we infer all
annotations), but fails to find a solution for six programs. In
contract, the unsound inhaling finds a solution for all examples, but
infers all annotations in only one case.

The best results are obtained through the unsound approach: in twelve
cases we infer the complete annotation, while in the remaining two
cases (\statement{AssociationList} and \statement{HandOverHand}), we
obtain an infeasible system.  These programs create difficulties for
all analyses because they traverse recursive data structures in
while-loops. These difficulties are caused by the imprecision of our
heap analysis, not by the permission inference. Therefore, we expect
that applying a more precise heap analysis will lead to better
results on these programs as well.

\goup
\section{Related Work}
\goup
\label{sect:relatedwork}

Access permissions are used in several verification frameworks such as
separation logic \cite{REY02} and implicit dynamic frames \cite{SJP09}
to reason about framing and concurrency. Fractional permissions
\cite{Boyland03} and counting permissions \cite{BornatCOP05} allow
permissions to be split in order to permit concurrent read access to a
resource.

The Chalice language \cite{LM09}, which we use in this paper, combines
implicit dynamic frames with fractional permissions and a flavor of
counting permissions, encoded through infinitesimal permissions. It
generally requires manual annotations, but provides an option
\texttt{-autoMagic} to infer certain permission predicates. However,
the inference does not find non-trivial splittings of permissions as
required by our running example.

There is a large body of work on the inference of program
annotations. For instance, Ernst et al.'s Daikon system \cite{EPG07}
uses a dynamic analysis to infer object invariants. Flanagan and
Leino's Houdini tool \cite{FL01} generates a large number of candidate
annotations and uses ESC/Java to verify or refute each of these
annotations. Leino and Logozzo \cite{LL05} integrate abstract
interpretation and program verification to infer loop
invariants. However, none of these inferences supports access
permissions.

A major application of access permissions is to simplify framing, that
is, determining what is definitely not changed by a method
execution. There are several static analyses for frame information.
Rakamari\'{c} and Hu \cite{RH08} propose a technique to infer frame
information for functions and loops on C programs.  Spoto and Poll
\cite{SP03} introduce a static analysis based on abstract
interpretation for JML's \statement{assignable} clauses. However,
their analysis only checks existing annotations, rather than inferring
annotations.  Cata{\~n}o and Huisman's Chase tool \cite{CH03} performs
similar checks. The practical effectiveness of their approach has been
demonstrated both in terms of precision and efficiency using
industrial code. However, the approach is not sound, since it does not
consider aliasing. In contrast to these approaches, we infer access
permissions, which can then be used to infer framing information
\cite{SJP09}, but also for other purposes such as verifying concurrent
code.




%In this context, as far as we know our approach is the first automatic inference (i) of access permissions, (ii) on Chalice programs, and (iii) that deals with monitor invariants, (abstract predicates), loop invariants, and pre and post conditions all together.

\goup
\section{Conclusion}
\goup
\label{sect:conclusion}

We presented a new approach to infer access permissions, including
fractional and Chalice's flavor of counting permissions. Our approach
infers monitor invariants, pre- and postconditions, and loop
invariants. It also handles abstract predicates, but we omitted them
in the paper for brevity.

Our static analysis infers a symbolic access permission for each
abstract location for each program point. It then constructs a
constraint system over these symbolic permissions that reflects the
permission requirements of Chalice.  The constraints are solved using
linear programming, obtaining numerical access permissions that are
then translated into program annotations. We implemented our approach
in \Sample. The experimental results indicate that our analysis is
efficient and precise.

As future work, we plan to integrate a shape analysis \cite{LS00} into
\Sample\ to obtain more precise heap abstractions and, thus, better
permission inference.

%\noindent \textbf{Future work} First of all, we plan to apply our approach using more refined heap analyses, and in particular shape analysis. We are currently working on plugging TVLA \cite{LS00} in \Sample. In addition, we want to apply our approach to bigger programs and study the experimental results on them.

\bibliographystyle{abbrv}
\goup
\bibliography{bibliografia}

%\newpage
%
%\appendix
%
%\section{Heap analysis}
%\label{appendix:heapanalysis}
%
%\subsection{Approximating the entry state of methods}
%When we start the analysis of a method, we do not have information on the heap structure. A trivial and too much approximated approach would be to consider that all the arguments may be aliases of the same concrete reference. Instead, our approach relies on type information in order to discover which arguments cannot be aliases. If the types of two arguments are compatible (that is, they are the same or one is subtype of the other), they may alias the same reference. Then we create an abstract node that may represent many concrete references, and we set the two arguments to point to it. We apply this approach both to the method's arguments and to the fields of the current object.\\
%\textbf{Formal definition:} Let $\arefarguments=\{\ael{a}_1, \cdots, \ael{a}_n\}$ be the set containing the abstract nodes created to represent the nodes at the beginning of the analysis of a method. Each element corresponds to an argument of the method or a field of the current class. Let $\statement{c_0}, \statement{c_1}, ..., \statement{c_n}$ be the list of the arguments of the method we are going to analyze and of the fields of the current class. The initial abstract state $\ael{\sigma}_0$ is defined as follows:
%\[
%\ael{\sigma}_0=\{ [\statement{c_i} \mapsto \ael{a}] : i \in [1..n], \ael{a} = \left\{
%\begin{array}{ll}
%\ael{a}_i & \textrm{ if } \exists j \in [1..i-1] : j=\min(\{z : \\
%& \cfunction{type}(\statement{c_z}) \leq \cfunction{type}(\statement{c_i}) \lor \cfunction{type}(\statement{c_i}) \leq \cfunction{type}(\statement{c_z})\})\\
%\ael{a}_j & \textrm{ otherwise}\\
%\end{array}
%\right.
%\}
%\]
%
%In addition, for each method call \statement{x.m()} contained in the program we analyze, we suppose to have a set $\arefarguments_{\statement{x.m()}}=\{\ael{r}_1, \cdots, \ael{r}_k\}$ that contains an abstract node for each access permission specified in the preconditions of method \statement{m}.\\
%
%\subsection{Approximating memory created inside the method}
%A method may potentially create an unbounded number of references at runtime (for instance, inside a loop). In order to obtain a computable approximation of the heap structure, we need to build up a finite approximation of concrete heaps.
%
%Our approach is to abstract each reference created inside the method we are analyzing with the program point that creates it. Since the length of the program is bounded, also the number of program points of a method is bounded. Hence, our approach contains a finite number of abstract node. As obvious consequence, this means that some abstract nodes would represent many concrete references.\\
%\textbf{Formal definition:} Let \programpoints\ be the set containing all the program points. The set of abstract nodes representing references created inside the analyzed method is defined as $\arefinside=\programpoints$.\\
%
%\subsection{Accesses to fields of objects}
%We represent fields of objects as pairs composed by the abstract address of the object, and the field's name. In this way we precisely represent each field of an object with a distinct identifier. This is particularly important for our analysis since we want to infer access permissions.\\
%\textbf{Formal definition:} Let \fields\ be the set of fields' names. We represent nodes to fields of object by elements in $\areffields=\pair{(\arefarguments \cup \arefinside)}{\fields}$. $(\ael{a}, \statement{f})$ represents an access to field \statement{f} on the abstract node \ael{a}.
%
%\subsection{Boxed domain with sets of abstract nodes}
%At a given point of a program objects created by different program points or accessed using different arguments or fields of the current object may be assigned to the same variable. Then our abstract domain collects sets of abstract nodes. The overall domain represents a state of the heap relating identifiers to abstract nodes. An identifier may be (i) a variable $\statement{x} \in \variables$, (ii) the identifier of a field $(\ael{a}, \statement{f}) \in \areffields$.\\
%\textbf{Formal definition:} Let $\aref=\arefarguments \cup \areffields \cup \arefinside$ be the set containing all the possible abstract nodes. The domain adopted to infer information about what variables may point is the powerset of this set. The lattice operators are the common ones on sets. Formally, $\langle \parts{\aref}, \subseteq, \emptyset, \aref, \cup, \cap \rangle$. A state of our abstract domain relates each identifier $\aid=\variables \cup \areffields$ to an element of this domain, that is $\aheap : \funzione{\aid}{\parts{\aref}}$.
%
%\subsection{Representing several concrete references}
%\label{sect:singleref}
%In order to recover some precision, we track if an abstract node represents one or many concrete references. If it represents one concrete reference, we can perform strong instead of weak updates \cite{RR99}, and we can inhale permissions on that abstract node, since we are sure to obtain permissions on a specific reference.
%
%Formally, a function $\afunction{\isSummary} :\funzione{\parts{\aref}}{\{\statement{true}, \statement{false}\}} $ is provided. It returns \statement{true} if and only if the given abstract node represents is a summary node. Intuitively, if $\ael{\epsilon} \in \parts{\aref}$ contains more than one abstract node, than $\aisSummary{\ael{\epsilon}}=\statement{true}$. Otherwise, it contains only one $\ael{r} \in \aref$. In this context, there are only three possible cases. In the first one, $\ael{r}$ was created inside the method, that is $\ael{r} \in \arefinside$. Then $\ael{r}=\cel{pp}$ where $\cel{pp}$ is a point of the program. Here $\aisSummary{\{\ael{r}\}}=\statement{true}$ if and only if $\cel{pp}$ is inside a loop. In the second case, $\ael{r}$ was created at the beginning of the analysis, that is $\ael{r} \in \arefarguments$. Here $\aisSummary{\{\ael{r}\}}=\statement{true}$ if and only if there is more than one argument or field that points to this abstract node at the beginning of the analysis of the method. In the last case, $\ael{r}$ is the access of a field, that is $\ael{r} \in \areffields$, then $\ael{r}=(\ael{r'}, \statement{field})$ where $\ael{r'} \in \arefinside \cup \arefarguments$. Here $\aisSummary{\{\ael{r}\}}=\aisSummary{\{\ael{r'}\}}$, where $\aisSummary{\{\ael{r'}\}}$ is computed following one of the two previous definitions.
%
%\subsection{Semantics}
%Before defining the semantics of statements, we introduce the function $\afunction{\semanticanome{E}}$ that, given a field access or the instantiation of an object and a state of the heap analysis, returns a set of abstract nodes. In the definition of the semantics, we will denote by \statements\ the set of statements.
%\[
%\begin{array}{l}
%\asemantics{E}{\statement{x.f}, \ael{h}}=\{(\ael{o}, \statement{f}) : \ael{o} \in \ael{h}(\ael{\statement{x}})\}\\
%\asemantics{E}{\statement{new\ T()}, \ael{h}}=\{\statement{pp} : \statement{pp} \textrm{ is the program point of } \statement{new\ T()}\}\\
%\end{array}
%\]
%\[
%\begin{array}{l}
%\afunction{\semanticanome{H}} : \funzione{(\pair{\statements}{\aheap})}{\aheap}\\
%
%\asemantics{H}{\statement{x = y.f}, \ael{h}}= \ael{h}[\statement{x} \mapsto \asemantics{E}{\statement{y.f}, \ael{h}}]\\
%
%\asemantics{H}{\statement{x = new\ T}, \ael{h}}= \ael{h}[\statement{x} \mapsto \asemantics{E}{\statement{new\ T}, \ael{h}}]\\
%
%\asemantics{H}{\statement{x.f = y.f}, \ael{h}}=\\
%\hspace{10pt} =\ael{h}[\ael{a} \mapsto 
%\left\{
%\begin{array}{ll}
%\asemantics{E}{\statement{y.f}, \ael{h}} & \textrm{ if } \aisSummary{\asemantics{E}{\statement{x.f}, \ael{h}}}=\statement{false}\\
%\asemantics{E}{\statement{y.f}, \ael{h}} \cup \ael{h}(\ael{a}) & \textrm{ if } \aisSummary{\asemantics{E}{\statement{x.f}, \ael{h}}}=\statement{true}\\
%\end{array}
%\right.
% : \ael{a} \in \asemantics{E}{\statement{x.f}, \ael{h}}]\\
%
%\asemantics{H}{\statement{x.f = new\ T}, \ael{h}}=\\
%\hspace{10pt} =\ael{h}[\ael{a} \mapsto \left\{
%\begin{array}{ll}
%\asemantics{E}{\statement{new\ T}, \ael{h}} & \textrm{ if } \aisSummary{\asemantics{E}{\statement{x.f}, \ael{h}}}=\statement{false}\\
%\asemantics{E}{\statement{new\ T}, \ael{h}} \cup \ael{h}(\ael{a}) & \textrm{ if } \aisSummary{\asemantics{E}{\statement{x.f}, \ael{h}}}=\statement{true}\\
%\end{array}
%\right.
% : \ael{a} \in \asemantics{E}{\statement{x.f}, \ael{h}}]\\
%
%\asemantics{H}{\statement{T\ x}, \ael{h}}=\ael{h}\\
%
%\asemantics{H}{\statement{x.m()}, \ael{h}}=\ael{h}[\ael{a}_i \mapsto \atopaddress{\agettype{\ael{a}_i}}{\dom{\ael{h}} \cup \arefarguments_{\statement{x.m()}}} :\\
%\hspace{30pt} \aprecondition{\ael{h}, \statement{x.m()}}=\{(\ael{a}_1, \ael{l}_1), \cdots, (\ael{a}_1, \ael{l}_1)\} \land \arefarguments_{\statement{x.m()}}=\{\ael{r}_1, \cdots, \ael{r}_n \} \subseteq \arefarguments \land\\
%\hspace{30pt} \forall \cel{i} \in [1..\cel{n}] : \agettype{\ael{r}_i}=\agettype{\ael{a}_i}]\\
%\hspace{30pt} \textrm{where } \atopaddress{\cel{T}}{\aset{R}}=\{\ael{a} \in \aset{R} : \agettype{\ael{a}} \leq \ael{T}\}
%
%\end{array}
%\]
%The semantics of the heap analysis tracks the modifications of the heap structure performed by different statements. When assigning local variables, we simply infer that the local variable points to the new abstract nodes after the assignment. Instead, when assigning the field of an object, we have to take into account if the abstract represents one or many concrete references through the $\afunction{\isSummary}$ function. In the first case we can perform strong updates, while in the latter one we have to perform weak updates losing some precision. The analysis of method calls exploits the fact that permissions specify frame information \cite{LMS09}. Intuitively, if a method does not require access to a heap location, then it cannot modify it. Then we set to top all the heap locations that can be accessed by the method in order to preserve the soundness. This corresponds to assign all the nodes that are compatible to the accessible location (that is, all the nodes whose type is the same or a subtype of the assigned location). This set of nodes is composed by all the existing nodes plus a set of nodes that could have been created inside the method and that are compatible with the types of the accessible locations.

\end{document}



