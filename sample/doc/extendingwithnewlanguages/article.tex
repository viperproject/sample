% This Source Code Form is subject to the terms of the Mozilla Public
% License, v. 2.0. If a copy of the MPL was not distributed with this
% file, You can obtain one at http://mozilla.org/MPL/2.0/.
%
% Copyright (c) 2011-2019 ETH Zurich.

\documentclass[11pt]{article}
\oddsidemargin 0.0cm
\evensidemargin 0.0cm
\textwidth 16.0cm
\headheight 0.0cm
\topmargin 0.0cm
\textheight 19.0cm

\usepackage[final]{listings}

\lstset{language={Java},mathescape=true,flexiblecolumns=true,basicstyle=\sffamily\small,numberstyle=\footnotesize,stepnumber=1,numbersep=2pt}

\newcommand{\ScalaAnalyzer}{\ensuremath{\mathsf{Sample}}}
\newcommand{\statement}[1]{\ensuremath{\mathsf{#1}}}
\newcommand{\Java}{\ensuremath{\mathsf{Java}}}
\newcommand{\Scala}{\ensuremath{\mathsf{Scala}}}

%\renewcommand{\familydefault}{\sfdefault}
\begin{document}
\title{How to extend \ScalaAnalyzer\ to a new language}

\author{
Pietro Ferrara\\
ETH Zurich\\
\texttt{pietro.ferrara@inf.ethz.ch}
}

\maketitle

\begin{abstract}
\ScalaAnalyzer\ works on a simple language aimed at being expressive in order to be applied to all the modern object oriented programming languages. This document explains the main components of this simple language, and how to translate a language into this representation.
\end{abstract}

\section{Introduction}
Package \statement{OORepresentation} contains all the classes that have to be extended and instantiated in order to apply \ScalaAnalyzer\ to a language. \ScalaAnalyzer\ is implemented in \Scala. Thus you can easily extend it to other languages using \Scala\ but also \Java, as \Scala\ is compiled into \Java\ bytecode and it is quite easy to interface \Java\ code with \Scala\ libraries.

All the components introduced in this document has been implemented and commented in Scala. The API of \ScalaAnalyzer\ contains all the documentation on them. In particular, you can care only about \statement{OORepresentation} package.

\section{Identifiers, Program Points and Types}
Three identifiers, i.e. \statement{PackageIdentifier}, \statement{ClassIdentifier}, and \statement{MethodIdentifier}, must be implemented. They are intended to represent identifiers of packages, classes, and methods (usually a string containing its name). You have to take care to implement in particular methods \statement{equals} and \statement{hashCode} in order to semantically represent when two identifiers are equals.\\
\statement{ProgramPoint} represents univocally a point of the program (e.g. the line, column and class to which such program point belongs).\\
\statement{Type} is intended to represent the type hierarchy. It requires to implement all the methods of lattice structure (e.g. \statement{lub}) that represents the type hierarchy. In addition, it requires to implement some methods (i.e. \statement{isObject()}, \statement{isNumericalType()}, \statement{isStatic()}, \statement{getName()}, \statement{isBottomExcluding(types : Set[T])}) required in order to extract some information from the type system.

\section{Sequential Statements}
\statement{Statement} is extended by all the possible existing statements. There are essentially 8 main types of statements:
\begin{itemize}
\item \statement{Assignment} 
\item \statement{VariableDeclaration} 
\item \statement{Variable} 
\item \statement{FieldAccess} 
\item \statement{New} 
\item \statement{MethodCall}
\item \statement{NumericalConstant} 
\item \statement{Throw} 
\end{itemize}

\subsection{Statements as Expressions}
We choose to adopt a representation as much unified as possible while preserving the expressiveness of our language. Thus we do not distinguish between expressions and statements. The common intuition is that expressions return values (e.g. \statement{3+variable}) while statements do not (e.g. \statement{x=y+x;}). On the other hand, there may be some cases in which a statement is also an expression (e.g. a method call that returns a value). A generalization of such approach already adopted by some programming languages  (like \Scala) is to consider statements as a particular case of expressions that return an \textquotedblleft empty\textquotedblright\ value. We followed such approach when designing our simple language.

\subsection{Representing Arithmetic Expressions as Method Calls}
Usually, arithmetic expressions (as well as other actions like type castings) are natively defined by the programming language. This leads to more complex languages. Indeed, since we want to preserve the simplicity of our language without giving up in terms of expressiveness, we adopt a different solution following the approach of programming languages on which \emph{everything is an object} (e.g. \Scala). We represent arithmetic (and other) operators as method calls (e.g. \statement{3+2} is represented by \statement{3.+(2)}), and numerical constants as objects.

\subsection{Semantics of native methods}
Since we represent all the native operators of a programming language (e.g. arithmetic operators or dynamic type castings) through method calls we also provide a way to define and implement the semantics of such statements. \statement{NativeMethodSemantics} has to be implemented in order to define such semantics.

\section{Control Flow Graph}
The last type of statement is \statement{ControlFlowGraph}. We represent conditional jumps in it. A CFG is a weighted graph whose nodes are list of statements, and where the weight of edges can be a boolean value (meaning that the condition has to be evaluated to true or false in order to cross that edge), or nothing (in the case that it is not a conditional jump, that is, it represents a \statement{goto} statement). Thus each node can only have two weighted edges (one with true and the other one with false), or one not-weighted edge. In order to check a condition (usually represented by a method call, e.g. \statement{x.>=(0)}), we need to put it as last statement of the node, and then add edges weighted to true and false, starting from such node and pointing to the node that has been executed if the condition is evaluated to true and false.\\
Note that considering the CFG as a statement is particularly expressive. For instance, in this way we support statements like \statement{x=if(B)\ then\ 1\ else\ 2}.

\section{Method and Class structure}
In order to represent the structure of a class, we have three major components:
\begin{itemize}
\item \statement{MethodDeclaration}
\item \statement{FieldDeclaration}
\item \statement{ClassDefinition}
\end{itemize}

Fields and methods can be added to \statement{ClassDefinition} through \statement{addField} and \statement{addMethod} methods.


\section{Using Java}
The analyzer has been implemented in \Scala. Since \Scala\ is compiled into \Java\ bytecode, you can implement the translation from your language to our simple language using \Java. The interfaces will be as follows:
\begin{itemize}
\item \statement{PackageIdentifier}, \statement{MethodIdentifier}, and \statement{ClassIdentifier} are interfaces, \statement{ProgramPoint} is an abstract class, all of them do not require to implement any specific method;
\item \statement{Type} is an interface whose definition is as follows:

\begin{lstlisting}
interface Type<T extends Type<T>> extends Lattice<T>
{
  boolean isBottomExcluding(Set<T> paramSet);
  String getName();
  boolean isStatic();
  boolean isNumericalType();
  boolean isObject();
}
\end{lstlisting}

\item the eight types of sequential statements are implemented as classes whose constructor signatures are as follows

\begin{lstlisting}
public Assignment(ProgramPoint programpoint, Statement<T> left, Statement<T> right)
public VariableDeclaration(ProgramPoint programpoint, Variable<T> variable, T typ, 
                                                                   Statement<T> right)
public Variable(ProgramPoint programpoint, String id)
public FieldAccess(ProgramPoint pp, List<Statement<T>> objs, String field, T typ)
public New(ProgramPoint pp, T typ)
public MethodCall(ProgramPoint pp, Statement<T> method, List<T> parametricTypes, 
                                          List<Statement<T>> parameters, T returnedType)
public NumericalConstant(ProgramPoint pp, String value, T typ)
public Throw(ProgramPoint programpoint, Statement<T> expr)
\end{lstlisting}

Note that all these classes are parameterized on a generic type \statement{T} that is a subtype of interface \statement{Type}

\item \statement{ControlFlowGraph} is a class whose interesting methods are the following ones:
\begin{lstlisting}
  public void addEdge(int x1, int x2, Option x3)
  public void setNode(int x1, Object x2)
  public int addNode(Object x1)
\end{lstlisting}
Note that methods \statement{setNode} and \statement{addNode} receive as parameter an \statement{Object} as node of the CFG, but it is intended to be a list of statement, i.e. belonging to type \statement{List<Statement>}. This happens because of type erasure in \Java\ bytecode.

\item \statement{MethodDeclaration}, \statement{FieldDeclaration}, and \statement{ClassDefinition} are classes whose constructors are defined as for \Scala.

\item \statement{NativeMethodSemantics} is an interface that requires to implement the following method:
\begin{lstlisting}
public abstract <S extends State<S, T, E>, T extends Type<T>, E extends Value<T, E>> 
                                  S getSemantics(S paramS, MethodCall<T> paramMethodCall);
\end{lstlisting}

\end{itemize}


\end{document}



