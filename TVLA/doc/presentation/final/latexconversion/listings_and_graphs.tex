\documentclass[11pt,a4paper,english]{article}
\usepackage{hyperref}
\usepackage{multicol}
\usepackage{array}
\usepackage{url}
\usepackage{amsmath}
\usepackage{amssymb}
\usepackage{subfig}
\usepackage[english]{babel}
\usepackage{blindtext}
\usepackage{graphicx}
\usepackage{todonotes}
\usepackage{tikz}
\usepackage[section]{placeins}
\usepackage{afterpage}
\usetikzlibrary{arrows,positioning}
%\usepackage[disable]{todonotes}
\usepackage[a4paper, bottom=38mm, footskip=30mm]{geometry}


%%% Header and footer style
% TODO: improve header/footer styles
\usepackage{fancyhdr}
\pagestyle{fancy}
\fancyhf{}
\fancyhead[LE,RO]{\bfseries\thepage}
\fancyhead[LO]{\bfseries\rightmark}
\fancyhead[RE]{\bfseries\leftmark}
\renewcommand{\headrulewidth}{0.5pt}
\addtolength{\headheight}{0.5pt}
\fancypagestyle{plain}{%
   \fancyhf{}
   \fancyfoot[C]{\bfseries \thepage}
   \fancyhead{}%get rid of headers on plain pages
   \renewcommand{\headrulewidth}{0pt} % an the line
}




% enumeration formats
\renewcommand\theenumi{\arabic{enumi}}
\renewcommand\labelenumi{\theenumi.}
\renewcommand\theenumii{\roman{enumii}}
\renewcommand\labelenumii{\theenumii)}

% dont indent paragraphs 
\usepackage[parfill]{parskip}


%%% hyperlinks
\hypersetup{%
   pagebackref=true,
   colorlinks=true,
   bookmarks=true,
   bookmarksopen=false,
   bookmarksnumbered=true,
}


\usepackage{listings}

% Scala language definition
\lstdefinelanguage{scala}{
  morekeywords={abstract,case,catch,class,def,%
    do,else,extends,false,final,finally,%
    for,if,implicit,import,match,mixin,%
    new,null,object,override,package,%
    private,protected,requires,return,sealed,%
    super,this,throw,trait,true,try,%
    type,val,var,while,with,yield},
  otherkeywords={=>,<-,<\%,<:,>:,\#,@},
  sensitive=true,
  morecomment=[l]{//},
  morecomment=[n]{/*}{*/},
  morestring=[b]",
  morestring=[b]',
  morestring=[b]"""
}

% listing format for TVP
\lstdefinelanguage{tvp}{
  morekeywords={foreach, in, E,unique,function,nonabs},
  otherkeywords={\%\%,\%action,\%f,\%p,\%s},
  sensitive=true,
  morecomment=[l]{//},
  morecomment=[n]{/*}{*/},
  morestring=[b]",
  morestring=[b]',
  morestring=[b]"""
}

% listing format for TVS
\lstdefinelanguage{tvs}{
  morekeywords={},
  otherkeywords={\%n,\%p,->,:},
  sensitive=true,
  morecomment=[l]{//},
  morecomment=[n]{/*}{*/},
  morestring=[b]",
  morestring=[b]',
  morestring=[b]"""
}

\definecolor{dkgreen}{rgb}{0,0.6,0}
\definecolor{gray}{rgb}{0.5,0.5,0.5}
\definecolor{mauve}{rgb}{0.58,0,0.82}
	
% default listing format (Scala!)
\lstset{frame=tb,
  language=scala,
  aboveskip=3mm,
  belowskip=3mm,
  showstringspaces=false,
  columns=fixed,
  basicstyle={\small\ttfamily},
  numbers=left,
  numberstyle=\tiny\color{gray},
  keywordstyle=\color{blue},
  commentstyle=\color{dkgreen},
  stringstyle=\color{mauve},
  frame=single,
  breaklines=true,
  breakatwhitespace=true,
  tabsize=3,
  captionpos=b,
  numberbychapter=true
}



% FONTS
\usepackage{fontspec}
\usepackage{xcolor}
\usepackage{xunicode}
\usepackage{xltxtra}
\defaultfontfeatures{Mapping=tex-text}
\setromanfont{Palatino LT Std}
\setmonofont[Scale=0.85]{Luxi Mono}
\definecolor{HeadingColor}{HTML}{00335b}
\newfontfamily\titlefont[Color=HeadingColor]{Linux Libertine}

% TikZ heap graph styles
\tikzset{
  every node/.style={on grid,node distance=2cm and 2cm},
  hid/.style={circle,draw=blue!80,fill=blue!30,thick,minimum size=1cm},
  nn/.style={minimum size=1cm},
  sum/.style={circle,draw=black!80,fill=blue!30,thick,minimum size=1cm,dashed},
  var/.style={rectangle,draw=black,fill=black!20,thick,minimum size=0.6cm},
  %edge styles
  every edge/.style={->,draw=black,thick,>=latex}
}

% custom autoref names
\def\subsubsectionautorefname{Section}
\def\subsectionautorefname{Section}
\def\subfigureautorefname{Figure}
\def\sectionautorefname{Section}


% boxes for listings as subfigures
\newsavebox{\mylistingbox}
\newsavebox{\mygraphboxA}
\newsavebox{\mygraphboxB}
\newsavebox{\mygraphboxC}

% remove??
\afterpage{\clearpage}
\usepackage[inactive,tightpage,xetex]{preview}
%\PreviewEnvironment{lstlisting}
\DeclareMathSizes{11}{30}{40}{40}  % For size 12 text
\begin{document}
\thispagestyle{empty}

\begin{minipage}{.6\linewidth}%
\begin{lstlisting}[caption={List traversal testcase},label={lst:traverseList}]
def traverseSummarizedList(x: AcyclicList) = {
// variable which afterwards references 
// the last element
var end : AcyclicList = null

var cur = x
while (cur != null) {
end = cur
cur = cur.n
} 	  
}
\end{lstlisting}
\end{minipage}

\begin{tikzpicture}
  \node[hid] (a) {a};
  \node[var] (x) [left=of a] {x}
    edge [->] (a);
    
\end{tikzpicture}

\begin{tikzpicture}
  \node[hid] (a) {a};
  \node[hid] (b) [right=of a] {b}
    edge[<-] node[auto,swap] {f} (a);
 
\end{tikzpicture}



\begin{minipage}{.33\linewidth}%
\begin{lstlisting}[caption={List traversal testcase},label={prepend}]
def createList(n: Int) = {
  var x = new E
  var t: E = null
  
  var i = 1
  while (i < n) {
    t = new E
    t.f = x
    x = t
    t = null
    i += 1
  }
}
\end{lstlisting}
\end{minipage}


\begin{tikzpicture}
    \node[hid] (n1) {n1};
    \node[var] (x) [left=of n1] {x}
      edge [->] (n1);
  \end{tikzpicture}  
  

  \begin{tikzpicture}
    \node[hid] (n1) {n1};
    \node[var] (x) [left=of n1] {x}
      edge [->] (n1);
      
    \node[sum] (n2) [right=of n1] {n2}
    edge[<-,dashed] node[auto] {n} (n1)
    edge[->,dashed,loop] node[auto,swap] {n} (n2);

  \end{tikzpicture}

\begin{minipage}{.4\linewidth}%
\begin{lstlisting}[caption={wbla},label={cond}]
def iAssign(unknown:Boolean) = {
  val x = new IntNode
  
  if (unknown) {
    x.i = 1
  } else {
    x.i = 2
  }
}
\end{lstlisting}
\end{minipage}

\begin{tikzpicture}
    \node[hid] (n1) {n1};
    \node[var] (x) [left=of n1] {x}
      edge [->] (n1);
    \node[hid] (n2) [right=of n1] {n2}
      edge [<-] node[auto,swap] {i} (n1);
\end{tikzpicture}  


\begin{minipage}{.5\linewidth}%
\begin{lstlisting}[caption={Sum list elements
  testcase},label={lst:sumElements}]
def sumListElementsZero = {
  x = /* code to build and initialize 
         a list with fields set to 0 */

  var cur = x
  var sum = x.i

  while (cur != null) {
    sum += cur.i
    cur = cur.n
  } 	  
}
\end{lstlisting}
\end{minipage}

\begin{figure}
  \begin{center}

    \subfloat[Structure 1]{
  \begin{tikzpicture}
  \node[hid] (n1) {n1};
  \node[var] (x) [above=of n1] {x}
    edge [->] (n1);
  \node[hid] (n3) [below=of n1] {n3}
    edge [<-] node[auto] {i} (n1);

  \node[sum] (n2) [right=3cm of n1] {n2}
    edge [<-,dashed] node[auto] {n} (n1)
    edge [->,loop,dashed] node[auto,swap] {n} (n2);
  \node[sum] (n4) [below=of n2] {n4}
    edge [<-,dashed] node[auto] {i} (n2);
  \end{tikzpicture}}
  \subfloat[Heap IDs and semantic state]{
    \begin{tabular}[b]{|c|c|c|}
    \hline
    Abbr. & Full ID & Semantic Domain \\
    \hline
    \texttt{n1} & \texttt{L194C17\_0} & $\top$ \\
    \texttt{n2} & \texttt{L196C17\_0+L198C17\_0} & $\top$ \\
    \texttt{n3} & \texttt{L195C15\_0} & $[0..0]$ \\
    \texttt{n4} & \texttt{L197C15\_0+L199C15\_0} & $[0..0]$ \\
    \texttt{x} & \texttt{x} & $\top$ \\
    \texttt{sum} & \texttt{sum} & $[0..0]$ \\
    \hline
  \end{tabular}}
  
  \end{center}
  \caption{Summing up list elements: Result}
  \label{fig:zerosumResult}
\end{figure}

\begin{tikzpicture}
    \node[hid] (n1) {n1};
    \node[var] (x) [left=of n1] {x}
      edge [->] (n1);
    \node[var] (x) [left=of n1] {x}
      edge [->] (n1);
  \end{tikzpicture}  
  

\clearpage

% BIG EXAMPLE

%after first iteration
  \begin{tikzpicture}
  \node[hid] (n1) {n1};
  \node[var] (h) [above left=2cm and 1cm of n1] {h}
    edge [->] (n1);
  \node[var] (cur) [above right=2cm and 1cm of n1] {cur}
    edge [->] (n1);
  \node[hid] (n3) [below=of n1] {n2}
    edge [<-] node[auto] {i} (n1);
  \end{tikzpicture}

% after second iteration
\begin{tikzpicture}
  \node[hid] (n1) {n3};
  \node[var] (h) [above left=2cm and 1cm of n1] {h}
    edge [->] (n1);
  \node[var] (cur) [above right=2cm and 1cm of n1] {cur}
    edge [->] (n1);
  \node[hid] (n3) [below=of n1] {n4}
    edge [<-] node[auto] {i} (n1);

  \node[hid] (n2) [right=3cm of n1] {n1}
    edge [<-] node[auto] {n} (n1);
  \node[hid] (n4) [below=of n2] {n2}
    edge [<-] node[auto] {i} (n2);
  \end{tikzpicture} 


% at end of third iteration, before h = cur
\begin{tikzpicture}
  \node[hid] (n1) {n5};
  \node[var] (cur) [above=2cm of n1] {cur}
    edge [->] (n1);
  \node[hid] (n3) [below=of n1] {n6}
    edge [<-] node[auto] {i} (n1);

  \node[hid] (n5) [right=3cm of n1] {n3}
    edge [<-] node[auto] {n} (n1);
    
  \node[var] (h) [above=2cm of n5] {h}
    edge [->] (n5);
   \node[hid] (n6) [below=of n5] {n4}
    edge [<-] node[auto] {i} (n5); 
 
    \node[hid] (n2) [right=3cm of n5] {n1}
    edge [<-] node[auto] {n} (n5);
  \node[hid] (n4) [below=of n2] {n2}
    edge [<-] node[auto] {i} (n2);
  \end{tikzpicture} 

% at end of third iteration, before summarization
\begin{tikzpicture}
  \node[hid] (n1) {n5};
  \node[var] (cur) [above right=2cm and 1cm of n1] {cur}
    edge [->] (n1);
  \node[var] (h) [above left=2cm and 1cm of n1] {h}
    edge [->] (n1);   
  \node[hid] (n3) [below=of n1] {n6}
    edge [<-] node[auto] {i} (n1);

  \node[hid] (n5) [right=3cm of n1] {n3}
    edge [<-] node[auto] {n} (n1);
    
   \node[hid] (n6) [below=of n5] {n4}
    edge [<-] node[auto] {i} (n5); 
 
    \node[hid] (n2) [right=3cm of n5] {n1}
    edge [<-] node[auto] {n} (n5);
  \node[hid] (n4) [below=of n2] {n2}
    edge [<-] node[auto] {i} (n2);
  \end{tikzpicture} 

% after third and any further iterations
  \begin{tikzpicture}
  \node[hid] (n1) {n5};
  \node[var] (h) [above left=2cm and 1cm of n1] {h}
    edge [->] (n1);
  \node[var] (cur) [above right=2cm and 1cm of n1] {cur}
    edge [->] (n1);
  \node[hid] (n3) [below=of n1] {n6}
    edge [<-] node[auto] {i} (n1);

  \node[sum] (n2) [right=3cm of n1] {n1}
    edge [<-,dashed] node[auto] {n} (n1)
    edge [->,loop,dashed] node[auto,swap] {n} (n2);
  \node[sum] (n4) [below=of n2] {n2}
    edge [<-,dashed] node[auto] {i} (n2);
  \end{tikzpicture}

\begin{minipage}{0.4\linewidth}
\begin{lstlisting}
def createZeroList(n: Int) = {
  var h: IntNode = null
  var cur :IntNode = null
  var i = 0

  while (i < n) {
    cur = new IntNode
    cur.i = 0
    cur.n = h
    h = cur
    i += 1
  } 	  
}
\end{lstlisting}
\end{minipage}
\end{document}
